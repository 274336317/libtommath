\documentclass[synpaper]{book}
\usepackage{hyperref}
\usepackage{makeidx}
\usepackage{amssymb}
\usepackage{color}
\usepackage{alltt}
\usepackage{graphicx}
\usepackage{layout}
\usepackage{appendix}
\def\union{\cup}
\def\intersect{\cap}
\def\getsrandom{\stackrel{\rm R}{\gets}}
\def\cross{\times}
\def\cat{\hspace{0.5em} \| \hspace{0.5em}}
\def\catn{$\|$}
\def\divides{\hspace{0.3em} | \hspace{0.3em}}
\def\nequiv{\not\equiv}
\def\approx{\raisebox{0.2ex}{\mbox{\small $\sim$}}}
\def\lcm{{\rm lcm}}
\def\gcd{{\rm gcd}}
\def\log{{\rm log}}
\def\ord{{\rm ord}}
\def\abs{{\mathit abs}}
\def\rep{{\mathit rep}}
\def\mod{{\mathit\ mod\ }}
\renewcommand{\pmod}[1]{\ ({\rm mod\ }{#1})}
\newcommand{\floor}[1]{\left\lfloor{#1}\right\rfloor}
\newcommand{\ceil}[1]{\left\lceil{#1}\right\rceil}
\def\Or{{\rm\ or\ }}
\def\And{{\rm\ and\ }}
\def\iff{\hspace{1em}\Longleftrightarrow\hspace{1em}}
\def\implies{\Rightarrow}
\def\undefined{{\rm ``undefined"}}
\def\Proof{\vspace{1ex}\noindent {\bf Proof:}\hspace{1em}}
\let\oldphi\phi
\def\phi{\varphi}
\def\Pr{{\rm Pr}}
\newcommand{\str}[1]{{\mathbf{#1}}}
\def\F{{\mathbb F}}
\def\N{{\mathbb N}}
\def\Z{{\mathbb Z}}
\def\R{{\mathbb R}}
\def\C{{\mathbb C}}
\def\Q{{\mathbb Q}}
\definecolor{DGray}{gray}{0.5}
\newcommand{\emailaddr}[1]{\mbox{$<${#1}$>$}}
\def\twiddle{\raisebox{0.3ex}{\mbox{\tiny $\sim$}}}
\def\gap{\vspace{0.5ex}}
\makeindex
\begin{document}
\frontmatter
\pagestyle{empty}
\title{LibTomMath User Manual \\ v1.2.0}
\author{LibTom Projects \\ www.libtom.net}
\maketitle
This text, the library and the accompanying textbook are all hereby placed in the public domain.  This book has been
formatted for B5 [176x250] paper using the \LaTeX{} {\em book} macro package.

\vspace{10cm}

\begin{flushright}Open Source.  Open Academia.  Open Minds.

\mbox{ }
LibTom Projects

\& originally

Tom St Denis,

Ontario, Canada
\end{flushright}

\tableofcontents
\listoffigures
\mainmatter
\pagestyle{headings}
\chapter{Introduction}
\section{What is LibTomMath?}
LibTomMath is a library of source code which provides a series of efficient and carefully written functions for manipulating
large integer numbers.  It was written in portable ISO C source code so that it will build on any platform with a conforming
C compiler.

In a nutshell the library was written from scratch with verbose comments to help instruct computer science students how
to implement ``bignum'' math.  However, the resulting code has proven to be very useful.  It has been used by numerous
universities, commercial and open source software developers.  It has been used on a variety of platforms ranging from
Linux and Windows based x86 to ARM based Gameboys and PPC based MacOS machines.

\section{License}
As of the v0.25 the library source code has been placed in the public domain with every new release.  As of the v0.28
release the textbook ``Implementing Multiple Precision Arithmetic'' has been placed in the public domain with every new
release as well.  This textbook is meant to compliment the project by providing a more solid walkthrough of the development
algorithms used in the library.

Since both\footnote{Note that the MPI files under \texttt{mtest/} are copyrighted by Michael Fromberger.  They are not required to use LibTomMath.} are in the
public domain everyone is entitled to do with them as they see fit.

\section{Building LibTomMath}

LibTomMath is meant to be very ``GCC friendly'' as it comes with a makefile well suited for GCC.  However, the library will
also build in MSVC, Borland C out of the box.  For any other ISO C compiler a makefile will have to be made by the end
developer. Please consider to commit such a makefile to the LibTomMath developers, currently residing at
\url{http://github.com/libtom/libtommath}, if successfully done so.

Intel's C-compiler (ICC) is sufficiently compatible with GCC, at least the newer versions, to replace GCC for building the static and the shared library. Editing the makefiles is not needed, just set the shell variable \texttt{CC} as shown below.
\begin{alltt}
CC=/home/czurnieden/intel/bin/icc make
\end{alltt}

ICC does not know all options available for GCC and LibTomMath uses two diagnostics \texttt{-Wbad-function-cast} and \texttt{-Wcast-align} that are not supported by ICC resulting in the warnings:
\begin{alltt}
icc: command line warning #10148: option '-Wbad-function-cast' not supported
icc: command line warning #10148: option '-Wcast-align' not supported
\end{alltt}
It is possible to mute this ICC warning with the compiler flag \texttt{-diag-disable=10148}\footnote{It is not recommended to suppress warnings without a very good reason but there is no harm in doing so in this very special case.}.

\subsection{Static Libraries}
To build as a static library for GCC issue the following
\begin{alltt}
make
\end{alltt}

command.  This will build the library and archive the object files in ``libtommath.a''.  Now you link against
that and include ``tommath.h'' within your programs.  Alternatively to build with MSVC issue the following
\begin{alltt}
nmake -f makefile.msvc
\end{alltt}

This will build the library and archive the object files in ``tommath.lib''.  This has been tested with MSVC
version 6.00 with service pack 5.

To run a program to adapt the Toom-Cook cut-off values to your architecture type
\begin{alltt}
make tune
\end{alltt}
This will take some time.

\subsection{Shared Libraries}
\subsubsection{GNU based Operating Systems}
To build as a shared library for GCC issue the following
\begin{alltt}
make -f makefile.shared
\end{alltt}
This requires the ``libtool'' package (common on most Linux/BSD systems).  It will build LibTomMath as both shared
and static then install (by default) into /usr/lib as well as install the header files in /usr/include.  The shared
library (resource) will be called \texttt{libtommath.la} while the static library called \texttt{libtommath.a}.  Generally
you use libtool to link your application against the shared object.

To run a program to adapt the Toom-Cook cut-off values to your architecture type
\begin{alltt}
make -f makefile.shared tune
\end{alltt}
This will take some time.

\subsubsection{Microsoft Windows based Operating Systems}
There is limited support for making a ``DLL'' in windows via the \texttt{makefile.cygwin\_dll} makefile.  It requires
Cygwin to work with since it requires the auto-export/import functionality.  The resulting DLL and import library
\texttt{libtommath.dll.a} can be used to link LibTomMath dynamically to any Windows program using Cygwin.
\subsubsection{OpenBSD}
OpenBSD replaced some of their GNU-tools, especially \texttt{libtool} with their own, slightly different versions. To ease the workload of LibTomMath's developer team, only a static library can be build with the included \texttt{makefile.unix}.

The wrong \texttt{make} will result in errors like:
\begin{alltt}
*** Parse error in /home/user/GITHUB/libtommath: Need an operator in 'LIBNAME' )
*** Parse error: Need an operator in 'endif' (makefile.shared:8)
*** Parse error: Need an operator in 'CROSS_COMPILE' (makefile_include.mk:16)
*** Parse error: Need an operator in 'endif' (makefile_include.mk:18)
*** Parse error: Missing dependency operator (makefile_include.mk:22)
*** Parse error: Missing dependency operator (makefile_include.mk:23)
...
\end{alltt}
The wrong \texttt{libtool} will build it all fine but when it comes to the final linking fails with
\begin{alltt}
...
cc -I./ -Wall -Wsign-compare -Wextra -Wshadow -Wsystem-headers -Wdeclaration-afo...
cc -I./ -Wall -Wsign-compare -Wextra -Wshadow -Wsystem-headers -Wdeclaration-afo...
cc -I./ -Wall -Wsign-compare -Wextra -Wshadow -Wsystem-headers -Wdeclaration-afo...
libtool --mode=link --tag=CC cc  error.lo s_mp_invmod_fast.lo fast_mp_mo
libtool: link: cc error.lo s_mp_invmod_fast.lo s_mp_montgomery_reduce_fast0
error.lo: file not recognized: File format not recognized
cc: error: linker command failed with exit code 1 (use -v to see invocation)
Error while executing cc error.lo s_mp_invmod_fast.lo fast_mp_montgomery0
gmake: *** [makefile.shared:64: libtommath.la] Error 1
\end{alltt}

To build a shared library with OpenBSD\footnote{Tested with OpenBSD version 6.4} the GNU versions of \texttt{make} and \texttt{libtool} are needed.
\begin{alltt}
$ sudo pkg_add gmake libtool
\end{alltt}
At this time two versions of \texttt{libtool} are installed and both are named \texttt{libtool}, unfortunately but GNU \texttt{libtool} has been placed in \texttt{/usr/local/bin/} and the native version in \texttt{/usr/bin/}. The path might be different in other versions of OpenBSD but both programms differ in the output of \texttt{libtool --version}
\begin{alltt}
$ /usr/local/bin/libtool --version
libtool (GNU libtool) 2.4.2
Written by Gordon Matzigkeit <gord@gnu.ai.mit.edu>, 1996

Copyright (C) 2011 Free Software Foundation, Inc.
This is free software; see the source for copying conditions.  There is NO
warranty; not even for MERCHANTABILITY or FITNESS FOR A PARTICULAR PURPOSE.
$ libtool --version
libtool (not (GNU libtool)) 1.5.26
\end{alltt}

The shared library should build now with
\begin{alltt}
LIBTOOL="/usr/local/bin/libtool" gmake -f makefile.shared
\end{alltt}
You might need to run a \texttt{gmake -f makefile.shared clean} first.

\subsubsection{NetBSD}
NetBSD is not as strict as OpenBSD but still needs \texttt{gmake} to build the shared library. \texttt{libtool} may also not exist in a fresh install.
\begin{alltt}
pkg_add gmake libtool
\end{alltt}
Please check with \texttt{libtool --version} that installed libtool is indeed a GNU libtool.
Build the shared library by typing:
\begin{alltt}
gmake -f makefile.shared
\end{alltt}

\subsection{Testing}
To build the library and the test harness type

\begin{alltt}
make test
\end{alltt}

This will build the library, \texttt{test} and \texttt{mtest/mtest}.  The \texttt{test} program will accept test vectors and verify the
results.  \texttt{mtest/mtest} will generate test vectors using the MPI library by Michael Fromberger\footnote{A copy of MPI
is included in the package}.  Simply pipe \texttt{mtest} into \texttt{test} using

\begin{alltt}
mtest/mtest | test
\end{alltt}

If you do not have a \texttt{/dev/urandom} style RNG source you will have to write your own PRNG and simply pipe that into
\texttt{mtest}.  For example, if your PRNG program is called \texttt{myprng} simply invoke

\begin{alltt}
myprng | mtest/mtest | test
\end{alltt}

This will output a row of numbers that are increasing.  Each column is a different test (such as addition, multiplication, etc)
that is being performed.  The numbers represent how many times the test was invoked.  If an error is detected the program
will exit with a dump of the relevant numbers it was working with.

\section{Build Configuration}
LibTomMath can configured at build time in three phases we shall call ``depends'', ``tweaks'' and ``trims''.
Each phase changes how the library is built and they are applied one after another respectively.

To make the system more powerful you can tweak the build process.  Classes are defined in the file
\texttt{tommath\_superclass.h}.  By default, the symbol \texttt{LTM\_ALL} shall be defined which simply
instructs the system to build all of the functions.  This is how LibTomMath used to be packaged.  This will give you
access to every function LibTomMath offers.

However, there are cases where such a build is not optional.  For instance, you want to perform RSA operations.  You
don't need the vast majority of the library to perform these operations.  Aside from \texttt{LTM\_ALL} there is
another pre--defined class \texttt{SC\_RSA\_1} which works in conjunction with the RSA from LibTomCrypt.  Additional
classes can be defined base on the need of the user.

\subsection{Build Depends}
In the file \texttt{tommath\_class.h} you will see a large list of C ``defines'' followed by a series of ``ifdefs''
which further define symbols.  All of the symbols (technically they're macros $\ldots$) represent a given C source
file.  For instance, \texttt{MP\_ADD\_C} represents the file \texttt{mp\_add.c}.  When a define has been enabled the
function in the respective file will be compiled and linked into the library.  Accordingly when the define
is absent the file will not be compiled and not contribute any size to the library.

You will also note that the header \texttt{tommath\_class.h} is actually recursively included (it includes itself twice).
This is to help resolve as many dependencies as possible.  In the last pass the symbol \texttt{LTM\_LAST} will be defined.
This is useful for ``trims''.

\subsection{Build Tweaks}
A tweak is an algorithm ``alternative''.  For example, to provide tradeoffs (usually between size and space).
They can be enabled at any pass of the configuration phase.

\begin{small}
\begin{center}
\begin{tabular}{|l|l|}
\hline \textbf{Define} & \textbf{Purpose} \\
\hline MP\_DIV\_SMALL & Enables a slower, smaller and equally \\
                          & functional mp\_div() function \\
\hline
\end{tabular}
\end{center}
\end{small}

\subsection{Build Trims}
A trim is a manner of removing functionality from a function that is not required.  For instance, to perform
RSA cryptography you only require exponentiation with odd moduli so even moduli support can be safely removed.
Build trims are meant to be defined on the last pass of the configuration which means they are to be defined
only if \texttt{LTM\_LAST} has been defined.

\subsubsection{Moduli Related}
\begin{small}
\begin{center}
\begin{tabular}{|l|l|}
\hline \textbf{Restriction} & \textbf{Undefine} \\
\hline Exponentiation with odd moduli only & S\_MP\_EXPTMOD\_C \\
                                           & MP\_REDUCE\_C \\
                                           & MP\_REDUCE\_SETUP\_C \\
                                           & S\_MP\_MUL\_HIGH\_DIGS\_C \\
                                           & FAST\_S\_MP\_MUL\_HIGH\_DIGS\_C \\
\hline Exponentiation with random odd moduli & (The above plus the following) \\
                                           & MP\_REDUCE\_2K\_C \\
                                           & MP\_REDUCE\_2K\_SETUP\_C \\
                                           & MP\_REDUCE\_IS\_2K\_C \\
                                           & MP\_DR\_IS\_MODULUS\_C \\
                                           & MP\_DR\_REDUCE\_C \\
                                           & MP\_DR\_SETUP\_C \\
\hline Modular inverse odd moduli only     & MP\_INVMOD\_SLOW\_C \\
\hline Modular inverse (both, smaller/slower) & FAST\_MP\_INVMOD\_C \\
\hline
\end{tabular}
\end{center}
\end{small}

\subsubsection{Operand Size Related}
\begin{small}
\begin{center}
\begin{tabular}{|l|l|}
\hline \textbf{Restriction} & \textbf{Undefine} \\
\hline Moduli $\le 2560$ bits              & MP\_MONTGOMERY\_REDUCE\_C \\
                                           & S\_MP\_MUL\_DIGS\_C \\
                                           & S\_MP\_MUL\_HIGH\_DIGS\_C \\
                                           & S\_MP\_SQR\_C \\
\hline Polynomial Schmolynomial            & MP\_KARATSUBA\_MUL\_C \\
                                           & MP\_KARATSUBA\_SQR\_C \\
                                           & MP\_TOOM\_MUL\_C \\
                                           & MP\_TOOM\_SQR\_C \\

\hline
\end{tabular}
\end{center}
\end{small}


\section{Purpose of LibTomMath}
Unlike  GNU MP (GMP) Library, LIP, OpenSSL or various other commercial kits (Miracl), LibTomMath was not written with
bleeding edge performance in mind.  First and foremost LibTomMath was written to be entirely open.  Not only is the
source code public domain (unlike various other GPL/etc licensed code), not only is the code freely downloadable but the
source code is also accessible for computer science students attempting to learn ``BigNum'' or multiple precision
arithmetic techniques.

LibTomMath was written to be an instructive collection of source code.  This is why there are many comments, only one
function per source file and often I use a ``middle-road'' approach where I don't cut corners for an extra 2\% speed
increase.

Source code alone cannot really teach how the algorithms work which is why I also wrote a textbook that accompanies
the library (beat that!).

So you may be thinking ``should I use LibTomMath?'' and the answer is a definite maybe.  Let me tabulate what I think
are the pros and cons of LibTomMath by comparing it to the math routines from GnuPG\footnote{GnuPG v1.2.3 versus LibTomMath v0.28}.

\newpage\begin{figure}[h]
\begin{small}
\begin{center}
\begin{tabular}{|l|c|c|l|}
\hline \textbf{Criteria} & \textbf{Pro} & \textbf{Con} & \textbf{Notes} \\
\hline Few lines of code per file & X & & GnuPG $ = 300.9$, LibTomMath  $ = 71.97$ \\
\hline Commented function prototypes & X && GnuPG function names are cryptic. \\
\hline Speed && X & LibTomMath is slower.  \\
\hline Totally free & X & & GPL has unfavourable restrictions.\\
\hline Large function base & X & & GnuPG is barebones. \\
\hline Five modular reduction algorithms & X & & Faster modular exponentiation for a variety of moduli. \\
\hline Portable & X & & GnuPG requires configuration to build. \\
\hline
\end{tabular}
\end{center}
\end{small}
\caption{LibTomMath Valuation}
\end{figure}

It may seem odd to compare LibTomMath to GnuPG since the math in GnuPG is only a small portion of the entire application.
However, LibTomMath was written with cryptography in mind.  It provides essentially all of the functions a cryptosystem
would require when working with large integers.

So it may feel tempting to just rip the math code out of GnuPG (or GnuMP where it was taken from originally) in your
own application but I think there are reasons not to.  While LibTomMath is slower than libraries such as GnuMP it is
not normally significantly slower.  On x86 machines the difference is normally a factor of two when performing modular
exponentiations.  It depends largely on the processor, compiler and the moduli being used.

Essentially the only time you wouldn't use LibTomMath is when blazing speed is the primary concern.  However,
on the other side of the coin LibTomMath offers you a totally free (public domain) well structured math library
that is very flexible, complete and performs well in resource constrained environments.  Fast RSA for example can
be performed with as little as 8KB of ram for data (again depending on build options).

\chapter{Getting Started with LibTomMath}
\section{Building Programs}
In order to use LibTomMath you must include ``tommath.h'' and link against the appropriate library file (typically
libtommath.a).  There is no library initialization required and the entire library is thread safe.

\section{Return Codes}
There are five possible return codes a function may return.

\index{MP\_OKAY}\index{MP\_YES}\index{MP\_NO}\index{MP\_VAL}\index{MP\_MEM}\index{MP\_ITER}\index{MP\_BUF}
\begin{figure}[h!]
\begin{center}
\begin{small}
\begin{tabular}{|l|l|}
\hline \textbf{Code} & \textbf{Meaning} \\
\hline MP\_OKAY & The function succeeded. \\
\hline MP\_VAL  & The function input was invalid. \\
\hline MP\_MEM  & Heap memory exhausted. \\
\hline MP\_ITER  & Maximum iterations reached. \\
\hline MP\_BUF  & Buffer overflow, supplied buffer too small.\\
\hline &\\
\hline MP\_YES  & Response is yes. \\
\hline MP\_NO   & Response is no. \\
\hline
\end{tabular}
\end{small}
\end{center}
\caption{Return Codes}
\end{figure}

The error codes \texttt{MP\_OKAY},\texttt{MP\_VAL}, \texttt{MP\_MEM}, \texttt{MP\_ITER}, and \texttt{MP\_BUF} are of the type \texttt{mp\_err}, the codes \texttt{MP\_YES} and \texttt{MP\_NO} are of type \texttt{mp\_bool}.

The last two codes listed are not actually ``return'ed'' by a function.  They are placed in an integer (the caller must
provide the address of an integer it can store to) which the caller can access.  To convert one of the three return codes
to a string use the following function.

\index{mp\_error\_to\_string}
\begin{alltt}
char *mp_error_to_string(int code);
\end{alltt}

This will return a pointer to a string which describes the given error code.  It will not work for the return codes \texttt{MP\_YES} and \texttt{MP\_NO}.

\section{Data Types}
The basic ``multiple precision integer'' type is known as the \texttt{mp\_int} within LibTomMath. This data type is used to
organize all of the data required to manipulate the integer it represents.  Within LibTomMath it has been prototyped
as the following.

\index{mp\_int}
\begin{alltt}
typedef struct  \{
   int used, alloc;
   mp_sign sign;
   mp_digit *dp;
\} mp_int;
\end{alltt}

Where \texttt{mp\_digit} is a data type that represents individual digits of the integer.  By default, an \texttt{mp\_digit} is the
ISO C \texttt{unsigned long} data type and each digit is $28-$bits long.  The \texttt{mp\_digit} type can be configured to suit other
platforms by defining the appropriate macros.

All LTM functions that use the \texttt{mp\_int} type will expect a pointer to \texttt{mp\_int} structure.  You must allocate memory to
hold the structure itself by yourself (whether off stack or heap it doesn't matter).  The very first thing that must be
done to use an \texttt{mp\_int} is that it must be initialized.

\section{Function Organization}

The arithmetic functions of the library are all organized to have the same style prototype.  That is source operands
are passed on the left and the destination is on the right.  For instance,

\begin{alltt}
mp_add(&a, &b, &c);       /* c = a + b */
mp_mul(&a, &a, &c);       /* c = a * a */
mp_div(&a, &b, &c, &d);   /* c = [a/b], d = a mod b */
\end{alltt}

Another feature of the way the functions have been implemented is that source operands can be destination operands as well.
For instance,

\begin{alltt}
mp_add(&a, &b, &b);       /* b = a + b */
mp_div(&a, &b, &a, &c);   /* a = [a/b], c = a mod b */
\end{alltt}

This allows operands to be re-used which can make programming simpler.

\section{Initialization}
\subsection{Single Initialization}
A single \texttt{mp\_int} can be initialized with the \texttt{mp\_init} function.

\index{mp\_init}
\begin{alltt}
mp_err mp_init (mp_int *a);
\end{alltt}

This function expects a pointer to an \texttt{mp\_int} structure and will initialize the members of the structure so the \texttt{mp\_int}
represents the default integer which is zero.  If the functions returns \texttt{MP\_OKAY} then the \texttt{mp\_int} is ready to be used
by the other LibTomMath functions.

\begin{small}
\begin{alltt}
int main(void)
\{
   mp_int number;
   mp_err result;

   if ((result = mp_init(&number)) != MP_OKAY) \{
      printf("Error initializing the number.  \%s",
             mp_error_to_string(result));
      return EXIT_FAILURE;
   \}

   /* use the number */

   return EXIT_SUCCESS;
\}
\end{alltt}
\end{small}

\subsection{Single Free}
When you are finished with an \texttt{mp\_int} it is ideal to return the heap it used back to the system.  The following function
provides this functionality.

\index{mp\_clear}
\begin{alltt}
void mp_clear (mp_int *a);
\end{alltt}

The function expects a pointer to a previously initialized \texttt{mp\_int} structure and frees the heap it uses.  It sets the
pointer\footnote{The \texttt{dp} member.} within the \texttt{mp\_int} to \texttt{NULL} which is used to prevent double free situations.
Is is legal to call \texttt{mp\_clear} twice on the same \texttt{mp\_int} in a row.

\begin{small}
\begin{alltt}
int main(void)
\{
   mp_int number;
   mp_err result;

   if ((result = mp_init(&number)) != MP_OKAY) \{
      printf("Error initializing the number.  \%s",
             mp_error_to_string(result));
      return EXIT_FAILURE;
   \}

   /* use the number */

   /* We're done with it. */
   mp_clear(&number);

   return EXIT_SUCCESS;
\}
\end{alltt}
\end{small}

\subsection{Multiple Initializations}
Certain algorithms require more than one large integer.  In these instances it is ideal to initialize all of the \texttt{mp\_int}
variables in an ``all or nothing'' fashion.  That is, they are either all initialized successfully or they are all
not initialized.

The  \texttt{mp\_init\_multi} function provides this functionality.

\index{mp\_init\_multi} \index{mp\_clear\_multi}
\begin{alltt}
mp_err mp_init_multi(mp_int *mp, ...);
\end{alltt}

It accepts a \texttt{NULL} terminated list of pointers to \texttt{mp\_int} structures.  It will attempt to initialize them all
at once.  If the function returns \texttt{MP\_OKAY} then all of the \texttt{mp\_int} variables are ready to use, otherwise none of them
are available for use.  A complementary \texttt{mp\_clear\_multi} function allows multiple \texttt{mp\_int} variables to be free'd
from the heap at the same time.

\begin{small}
\begin{alltt}
int main(void)
\{
   mp_int num1, num2, num3;
   mp_err result;

   if ((result = mp_init_multi(&num1,
                               &num2,
                               &num3, NULL)) != MP\_OKAY) \{
      printf("Error initializing the numbers.  \%s",
             mp_error_to_string(result));
      return EXIT_FAILURE;
   \}

   /* use the numbers */

   /* We're done with them. */
   mp_clear_multi(&num1, &num2, &num3, NULL);

   return EXIT_SUCCESS;
\}
\end{alltt}
\end{small}

\subsection{Other Initializers}
To initialized and make a copy of an \texttt{mp\_int} the \texttt{mp\_init\_copy} function has been provided.

\index{mp\_init\_copy}
\begin{alltt}
mp_err mp_init_copy (mp_int *a, mp_int *b);
\end{alltt}

This function will initialize $a$ and make it a copy of $b$ if all goes well.

\begin{small}
\begin{alltt}
int main(void)
\{
   mp_int num1, num2;
   mp_err result;

   /* initialize and do work on num1 ... */

   /* We want a copy of num1 in num2 now */
   if ((result = mp_init_copy(&num2, &num1)) != MP_OKAY) \{
     printf("Error initializing the copy.  \%s",
             mp_error_to_string(result));
      return EXIT_FAILURE;
   \}

   /* now num2 is ready and contains a copy of num1 */

   /* We're done with them. */
   mp_clear_multi(&num1, &num2, NULL);

   return EXIT_SUCCESS;
\}
\end{alltt}
\end{small}

Another less common initializer is \texttt{mp\_init\_size} which allows the user to initialize an \texttt{mp\_int} with a given
default number of digits.  By default, all initializers allocate \texttt{MP\_PREC} digits.  This function lets
you override this behaviour.

\index{mp\_init\_size}
\begin{alltt}
mp_err mp_init_size (mp_int *a, int size);
\end{alltt}

The $size$ parameter must be greater than zero.  If the function succeeds the \texttt{mp\_int} $a$ will be initialized
to have $size$ digits (which are all initially zero).

\begin{small}
\begin{alltt}
int main(void)
\{
   mp_int number;
   mp_err result;

   /* we need a 60-digit number */
   if ((result = mp_init_size(&number, 60)) != MP_OKAY) \{
      printf("Error initializing the number.  \%s",
             mp_error_to_string(result));
      return EXIT_FAILURE;
   \}

   /* use the number */

   return EXIT_SUCCESS;
\}
\end{alltt}
\end{small}

\section{Maintenance Functions}
\subsection{Clear Leading Zeros}

This is used to ensure that leading zero digits are trimmed and the leading "used" digit will be non-zero.
It also fixes the sign if there are no more leading digits.

\index{mp\_clamp}
\begin{alltt}
void mp_clamp(mp_int *a);
\end{alltt}

\subsection{Zero Out}

This function will set the ``bigint'' to zeros without changing the amount of allocated memory.

\index{mp\_zero}
\begin{alltt}
void mp_zero(mp_int *a);
\end{alltt}


\subsection{Reducing Memory Usage}
When an \texttt{mp\_int} is in a state where it won't be changed again\footnote{A Diffie-Hellman modulus for instance.} excess
digits can be removed to return memory to the heap with the \texttt{mp\_shrink} function.

\index{mp\_shrink}
\begin{alltt}
mp_err mp_shrink (mp_int *a);
\end{alltt}

This will remove excess digits of the \texttt{mp\_int} $a$.  If the operation fails the \texttt{mp\_int} should be intact without the
excess digits being removed.  Note that you can use a shrunk \texttt{mp\_int} in further computations, however, such operations
will require heap operations which can be slow.  It is not ideal to shrink \texttt{mp\_int} variables that you will further
modify in the system (unless you are seriously low on memory).

\begin{small}
 \begin{alltt}
int main(void)
\{
   mp_int number;
   mp_err result;

   if ((result = mp_init(&number)) != MP_OKAY) \{
      printf("Error initializing the number.  \%s",
             mp_error_to_string(result));
      return EXIT_FAILURE;
   \}

   /* use the number [e.g. pre-computation]  */

   /* We're done with it for now. */
   if ((result = mp_shrink(&number)) != MP_OKAY) \{
      printf("Error shrinking the number.  \%s",
             mp_error_to_string(result));
      return EXIT_FAILURE;
   \}

   /* use it .... */


   /* we're done with it. */
   mp_clear(&number);

   return EXIT_SUCCESS;
\}
\end{alltt}
 \end{small}

\subsection{Adding additional digits}

Within the mp\_int structure are two parameters which control the limitations of the array of digits that represent
the integer the mp\_int is meant to equal.   The \texttt{used} parameter dictates how many digits are significant, that is,
contribute to the value of the mp\_int.  The \texttt{alloc} parameter dictates how many digits are currently available in
the array.  If you need to perform an operation that requires more digits you will have to mp\_grow() the mp\_int to
your desired size.

\index{mp\_grow}
\begin{alltt}
mp_err mp_grow (mp_int *a, int size);
\end{alltt}

This will grow the array of digits of $a$ to $size$.  If the \texttt{alloc} parameter is already bigger than
$size$ the function will not do anything.

\begin{small}
\begin{alltt}
int main(void)
\{
   mp_int number;
   mp_err result;

   if ((result = mp_init(&number)) != MP_OKAY) \{
      printf("Error initializing the number.  \%s",
             mp_error_to_string(result));
      return EXIT_FAILURE;
   \}

   /* use the number */

   /* We need to add 20 digits to the number  */
   if ((result = mp_grow(&number, number.alloc + 20)) != MP_OKAY) \{
      printf("Error growing the number.  \%s",
             mp_error_to_string(result));
      return EXIT_FAILURE;
   \}


   /* use the number */

   /* we're done with it. */
   mp_clear(&number);

   return EXIT_SUCCESS;
\}
\end{alltt}
\end{small}

\chapter{Basic Operations}
\section{Copying}

A so called ``deep copy'', where new memory is allocated and all contents of $a$ are copied verbatim into $b$ such that $b = a$ at the end.

\index{mp\_copy}
\begin{alltt}
mp_err mp_copy (const mp_int *a, mp_int *b);
\end{alltt}

You can also just swap $a$ and $b$. It does the normal pointer changing with a temporary pointer variable, just that you do not have to.

\index{mp\_exch}
\begin{alltt}
void mp_exch (mp_int *a, mp_int *b);
\end{alltt}

\section{Bit Counting}

To get the position of the lowest bit set (LSB, the Lowest Significant Bit; the number of bits which are zero before the first zero bit )

\index{mp\_cnt\_lsb}
\begin{alltt}
int mp_cnt_lsb(const mp_int *a);
\end{alltt}

To get the position of the highest bit set (MSB, the Most Significant Bit; the number of bits in the ``bignum'')

\index{mp\_count\_bits}
\begin{alltt}
int mp_count_bits(const mp_int *a);
\end{alltt}


\section{Small Constants}
Setting an \texttt{mp\_int} to a small constant is a relatively common operation.  To accommodate these instances there is a
small constant assignment function.  This function is used to set a single digit constant.
The reason for this function is efficiency.  Setting a single digit is quick but the
domain of a digit can change (it's always at least $0 \ldots 127$).

\subsection{Single Digit}

Setting a single digit can be accomplished with the following function.

\index{mp\_set}
\begin{alltt}
void mp_set (mp_int *a, mp_digit b);
\end{alltt}

This will zero the contents of $a$ and make it represent an integer equal to the value of $b$.  Note that this
function has a return type of \texttt{void}.  It cannot cause an error so it is safe to assume the function
succeeded.

\begin{small}
\begin{alltt}
int main(void)
\{
   mp_int number;
   mp_err result;

   if ((result = mp_init(&number)) != MP_OKAY) \{
      printf("Error initializing the number.  \%s",
             mp_error_to_string(result));
      return EXIT_FAILURE;
   \}

   /* set the number to 5 */
   mp_set(&number, 5);

   /* we're done with it. */
   mp_clear(&number);

   return EXIT_SUCCESS;
\}
\end{alltt}
\end{small}

\subsection{Int32 and Int64 Constants}

These functions can be used to set a constant with 32 or 64 bits.

\index{mp\_set\_i32} \index{mp\_set\_u32}
\index{mp\_set\_i64} \index{mp\_set\_u64}
\begin{alltt}
void mp_set_i32 (mp_int *a, int32_t b);
void mp_set_u32 (mp_int *a, uint32_t b);
void mp_set_i64 (mp_int *a, int64_t b);
void mp_set_u64 (mp_int *a, uint64_t b);
\end{alltt}

These functions assign the sign and value of the input $b$ to the big integer $a$.
The value can be obtained again by calling the following functions.

\index{mp\_get\_i32} \index{mp\_get\_u32} \index{mp\_get\_mag\_u32}
\index{mp\_get\_i64} \index{mp\_get\_u64} \index{mp\_get\_mag\_u64}
\begin{alltt}
int32_t mp_get_i32 (const mp_int *a);
uint32_t mp_get_u32 (const mp_int *a);
uint32_t mp_get_mag_u32 (const mp_int *a);
int64_t mp_get_i64 (const mp_int *a);
uint64_t mp_get_u64 (const mp_int *a);
uint64_t mp_get_mag_u64 (const mp_int *a);
\end{alltt}

These functions return the 32 or 64 least significant bits of $a$ respectively. The unsigned functions
return negative values in a twos complement representation. The absolute value or magnitude can be obtained using the \texttt{mp\_get\_mag*} functions.

\begin{small}
\begin{alltt}
int main(void)
\{
   mp_int number;
   mp_err result;

   if ((result = mp_init(&number)) != MP_OKAY) \{
      printf("Error initializing the number.  \%s",
             mp_error_to_string(result));
      return EXIT_FAILURE;
   \}

   /* set the number to 654321 (note this is bigger than 127) */
   mp_set_u32(&number, 654321);

   printf("number == \%" PRIi32 "\textbackslash{}n", mp_get_i32(&number));

   /* we're done with it. */
   mp_clear(&number);

   return EXIT_SUCCESS;
\}
\end{alltt}
\end{small}

This should output the following if the program succeeds.

\begin{alltt}
number == 654321
\end{alltt}

\subsection{Long Constants - platform dependant}

\index{mp\_set\_l} \index{mp\_set\_ul}
\begin{alltt}
void mp_set_l (mp_int *a, long b);
void mp_set_ul (mp_int *a, unsigned long b);
\end{alltt}

This will assign the value of the platform-dependent sized variable $b$ to the big integer $a$.

To retrieve the value, the following functions can be used.

\index{mp\_get\_l} \index{mp\_get\_ul} \index{mp\_get\_mag\_ul}
\begin{alltt}
long mp_get_l (const mp_int *a);
unsigned long mp_get_ul (const mp_int *a);
unsigned long mp_get_mag_ul (const mp_int *a);
\end{alltt}

This will return the least significant bits of the big integer $a$ that fit into the native data type \texttt{long}.

\subsection{Long Long Constants - platform dependant}

\index{mp\_set\_ll} \index{mp\_set\_ull}
\begin{alltt}
void mp_set_ll (mp_int *a, long long b);
void mp_set_ull (mp_int *a, unsigned long long b);
\end{alltt}

This will assign the value of the platform-dependent sized variable $b$ to the big integer $a$.

To retrieve the value, the following functions can be used.

\index{mp\_get\_ll}
\index{mp\_get\_ull}
\index{mp\_get\_mag\_ull}
\begin{alltt}
long long mp_get_ll (const mp_int *a);
unsigned long long mp_get_ull (const mp_int *a);
unsigned long long mp_get_mag_ull (const mp_int *a);
\end{alltt}

This will return the least significant bits of $a$ that fit into the native data type \texttt{long long}.

\subsection{Floating Point Constants - platform dependant}

\index{mp\_set\_double}
\begin{alltt}
mp_err mp_set_double(mp_int *a, double b);
\end{alltt}

If the platform supports the floating point data type \texttt{double} (binary64) this function will assign the integer part of \texttt{b} to the big integer $a$. This function will return \texttt{MP\_VAL} if \texttt{b} is \texttt{+/-inf} or \texttt{NaN}. 

To convert a big integer to a \texttt{double} use

\index{mp\_get\_double}
\begin{alltt}
double mp_get_double(const mp_int *a);
\end{alltt}


\subsection{Initialize and Setting Constants}
To both initialize and set small constants the following nine functions are available.
\index{mp\_init\_set} \index{mp\_init\_set\_int}
\begin{alltt}
mp_err mp_init_set (mp_int *a, mp_digit b);
mp_err mp_init_i32 (mp_int *a, int32_t b);
mp_err mp_init_u32 (mp_int *a, uint32_t b);
mp_err mp_init_i64 (mp_int *a, int64_t b);
mp_err mp_init_u64 (mp_int *a, uint64_t b);
mp_err mp_init_l   (mp_int *a, long b);
mp_err mp_init_ul  (mp_int *a, unsigned long b);
mp_err mp_init_ll  (mp_int *a, long long b);
mp_err mp_init_ull (mp_int *a, unsigned long long b);
\end{alltt}

Both functions work like the previous counterparts except they first initialize $a$ with the function \texttt{mp\_init} before setting the values.

\begin{alltt}
int main(void)
\{
   mp_int number1, number2;
   mp_err    result;

   /* initialize and set a single digit */
   if ((result = mp_init_set(&number1, 100)) != MP_OKAY) \{
      printf("Error setting number1: \%s",
             mp_error_to_string(result));
      return EXIT_FAILURE;
   \}

   /* initialize and set a long */
   if ((result = mp_init_l(&number2, 1023)) != MP_OKAY) \{
      printf("Error setting number2: \%s",
             mp_error_to_string(result));
      return EXIT_FAILURE;
   \}

   /* display */
   printf("Number1, Number2 == \%" PRIi32 ", \%" PRIi32 "\textbackslash{}n",
          mp_get_i32(&number1), mp_get_i32(&number2));

   /* clear */
   mp_clear_multi(&number1, &number2, NULL);

   return EXIT_SUCCESS;
\}
\end{alltt}

If this program succeeds it shall output.
\begin{alltt}
Number1, Number2 == 100, 1023
\end{alltt}



\section{Comparisons}

Comparisons in LibTomMath are always performed in a ``left to right'' fashion.  There are three possible return codes
for any comparison.

\index{MP\_GT} \index{MP\_EQ} \index{MP\_LT}
\begin{figure}[h]
\begin{center}
\begin{tabular}{|c|c|}
\hline \textbf{Result Code} & \textbf{Meaning} \\
\hline MP\_GT & $a > b$ \\
\hline MP\_EQ & $a = b$ \\
\hline MP\_LT & $a < b$ \\
\hline
\end{tabular}
\end{center}
\caption{Comparison Codes for $a, b$}
\label{fig:CMP}
\end{figure}

In figure \ref{fig:CMP} two integers $a$ and $b$ are being compared.  In this case $a$ is said to be ``to the left'' of
$b$. The return codes are of type \texttt{mp\_ord}.

\subsection{Unsigned comparison}

An unsigned comparison considers only the digits themselves and not the associated \texttt{sign} flag of the
\texttt{mp\_int} structures.  This is analogous to an absolute comparison.  The function \texttt{mp\_cmp\_mag} will compare two
\texttt{mp\_int} variables based on their digits only.

\index{mp\_cmp\_mag}
\begin{alltt}
mp_ord mp_cmp_mag(mp_int *a, mp_int *b);
\end{alltt}
This will compare $a$ to $b$ placing $a$ to the left of $b$.  This function cannot fail and will return one of the
three compare codes listed in figure \ref{fig:CMP}.

\begin{small}
\begin{alltt}
int main(void)
\{
   mp_int number1, number2;
   mp_err result;

   if ((result = mp_init_multi(&number1, &number2, NULL)) != MP_OKAY) \{
      printf("Error initializing the numbers.  \%s",
             mp_error_to_string(result));
      return EXIT_FAILURE;
   \}

   /* set the number1 to 5 */
   mp_set(&number1, 5);

   /* set the number2 to -6 */
   mp_set(&number2, 6);
   if ((result = mp_neg(&number2, &number2)) != MP_OKAY) \{
      printf("Error negating number2.  \%s",
             mp_error_to_string(result));
      return EXIT_FAILURE;
   \}

   switch(mp_cmp_mag(&number1, &number2)) \{
       case MP_GT:  printf("|number1| > |number2|"); break;
       case MP_EQ:  printf("|number1| = |number2|"); break;
       case MP_LT:  printf("|number1| < |number2|"); break;
   \}

   /* we're done with it. */
   mp_clear_multi(&number1, &number2, NULL);

   return EXIT_SUCCESS;
\}
\end{alltt}
\end{small}

If this program\footnote{This function uses the \texttt{mp\_neg} function which is discussed in section \ref{sec:NEG}.} completes
successfully it should print the following.

\begin{alltt}
|number1| < |number2|
\end{alltt}

This is because $\vert -6 \vert = 6$ and obviously $5 < 6$.

\subsection{Signed comparison}

To compare two \texttt{mp\_int} variables based on their signed value the \texttt{mp\_cmp} function is provided.

\index{mp\_cmp}
\begin{alltt}
mp_ord mp_cmp(mp_int *a, mp_int *b);
\end{alltt}

This will compare $a$ to the left of $b$.  It will first compare the signs of the two \texttt{mp\_int} variables.  If they
differ it will return immediately based on their signs.  If the signs are equal then it will compare the digits
individually.  This function will return one of the compare conditions codes listed in figure \ref{fig:CMP}.

\begin{small}
\begin{alltt}
int main(void)
\{
   mp_int number1, number2;
   mp_err result;

   if ((result = mp_init_multi(&number1, &number2, NULL)) != MP_OKAY) \{
      printf("Error initializing the numbers.  \%s",
             mp_error_to_string(result));
      return EXIT_FAILURE;
   \}

   /* set the number1 to 5 */
   mp_set(&number1, 5);

   /* set the number2 to -6 */
   mp_set(&number2, 6);
   if ((result = mp_neg(&number2, &number2)) != MP_OKAY) \{
      printf("Error negating number2.  \%s",
             mp_error_to_string(result));
      return EXIT_FAILURE;
   \}

   switch(mp_cmp(&number1, &number2)) \{
       case MP_GT:  printf("number1 > number2"); break;
       case MP_EQ:  printf("number1 = number2"); break;
       case MP_LT:  printf("number1 < number2"); break;
   \}

   /* we're done with it. */
   mp_clear_multi(&number1, &number2, NULL);

   return EXIT_SUCCESS;
\}
\end{alltt}
\end{small}

If this program\footnote{This function uses the \texttt{mp\_neg} function which is discussed in section \ref{sec:NEG}.} completes
successfully it should print the following.

\begin{alltt}
number1 > number2
\end{alltt}

\subsection{Single Digit}

To compare a single digit against an \texttt{mp\_int} the following function has been provided.

\index{mp\_cmp\_d}
\begin{alltt}
mp_ord mp_cmp_d(mp_int *a, mp_digit b);
\end{alltt}

This will compare $a$ to the left of $b$ using a signed comparison.  Note that it will always treat $b$ as
positive.  This function is rather handy when you have to compare against small values such as $1$ (which often
comes up in cryptography).  The function cannot fail and will return one of the tree compare condition codes
listed in figure \ref{fig:CMP}.


\begin{small}
\begin{alltt}
int main(void)
\{
   mp_int number;
   mp_err result;

   if ((result = mp_init(&number)) != MP_OKAY) \{
      printf("Error initializing the number.  \%s",
             mp_error_to_string(result));
      return EXIT_FAILURE;
   \}

   /* set the number to 5 */
   mp_set(&number, 5);

   switch(mp_cmp_d(&number, 7)) \{
       case MP_GT:  printf("number > 7"); break;
       case MP_EQ:  printf("number = 7"); break;
       case MP_LT:  printf("number < 7"); break;
   \}

   /* we're done with it. */
   mp_clear(&number);

   return EXIT_SUCCESS;
\}
\end{alltt}
\end{small}

If this program functions properly it will print out the following.

\begin{alltt}
number < 7
\end{alltt}

\section{Logical Operations}

Logical operations are operations that can be performed either with simple shifts or boolean operators such as
AND, XOR and OR directly.  These operations are very quick.

\subsection{Multiplication by two}

Multiplications and divisions by any power of two can be performed with quick logical shifts either left or
right depending on the operation.

When multiplying or dividing by two a special case routine can be used which are as follows.
\index{mp\_mul\_2} \index{mp\_div\_2}
\begin{alltt}
mp_err mp_mul_2(const mp_int *a, mp_int *b);
mp_err mp_div_2(const mp_int *a, mp_int *b);
\end{alltt}

The former will assign twice $a$ to $b$ while the latter will assign half $a$ to $b$.  These functions are fast
since the shift counts and maskes are hardcoded into the routines.

\begin{small}
\begin{alltt}
int main(void)
\{
   mp_int number;
   mp_err result;

   if ((result = mp_init(&number)) != MP_OKAY) \{
      printf("Error initializing the number.  \%s",
             mp_error_to_string(result));
      return EXIT_FAILURE;
   \}

   /* set the number to 5 */
   mp_set(&number, 5);

   /* multiply by two */
   if ((result = mp\_mul\_2(&number, &number)) != MP_OKAY) \{
      printf("Error multiplying the number.  \%s",
             mp_error_to_string(result));
      return EXIT_FAILURE;
   \}
   switch(mp_cmp_d(&number, 7)) \{
       case MP_GT:  printf("2*number > 7"); break;
       case MP_EQ:  printf("2*number = 7"); break;
       case MP_LT:  printf("2*number < 7"); break;
   \}

   /* now divide by two */
   if ((result = mp\_div\_2(&number, &number)) != MP_OKAY) \{
      printf("Error dividing the number.  \%s",
             mp_error_to_string(result));
      return EXIT_FAILURE;
   \}
   switch(mp_cmp_d(&number, 7)) \{
       case MP_GT:  printf("2*number/2 > 7"); break;
       case MP_EQ:  printf("2*number/2 = 7"); break;
       case MP_LT:  printf("2*number/2 < 7"); break;
   \}

   /* we're done with it. */
   mp_clear(&number);

   return EXIT_SUCCESS;
\}
\end{alltt}
\end{small}

If this program is successful it will print out the following text.

\begin{alltt}
2*number > 7
2*number/2 < 7
\end{alltt}

Since $10 > 7$ and $5 < 7$.

To multiply by a power of two the following function can be used.

\index{mp\_mul\_2d}
\begin{alltt}
mp_err mp_mul_2d(const mp_int *a, int b, mp_int *c);
\end{alltt}

This will multiply $a$ by $2^b$ and store the result in $c$.  If the value of $b$ is less than or equal to
zero the function will copy $a$ to $c$ without performing any further actions.  The multiplication itself
is implemented as a right-shift operation of $a$ by $b$ bits.

To divide by a power of two use the following.

\index{mp\_div\_2d}
\begin{alltt}
mp_err mp_div_2d (const mp_int *a, int b, mp_int *c, mp_int *d);
\end{alltt}
Which will divide $a$ by $2^b$, store the quotient in $c$ and the remainder in $d$.  If $b \le 0$ then the
function simply copies $a$ over to $c$ and zeroes $d$.  The variable $d$ may be passed as a \texttt{NULL}
value to signal that the remainder is not desired.  The division itself is implemented as a left-shift
operation of $a$ by $b$ bits.

It is also not very uncommon to need just the power of two $2^b$;  for example as a start-value for the Newton method.

\index{mp\_2expt}
\begin{alltt}
mp_err mp_2expt(mp_int *a, int b);
\end{alltt}
It is faster than doing it by shifting $1$ with \texttt{mp\_mul\_2d}.

\subsection{Polynomial Basis Operations}

Strictly speaking the organization of the integers within the mp\_int structures is what is known as a
``polynomial basis''.  This simply means a field element is stored by divisions of a radix.  For example, if
$f(x) = \sum_{i=0}^{k} y_ix^k$ for any vector $\vec y$ then the array of digits in $\vec y$ are said to be
the polynomial basis representation of $z$ if $f(\beta) = z$ for a given radix $\beta$.

To multiply by the polynomial $g(x) = x$ all you have todo is shift the digits of the basis left one place.  The
following function provides this operation.

\index{mp\_lshd}
\begin{alltt}
mp_err mp_lshd (mp_int *a, int b);
\end{alltt}

This will multiply $a$ in place by $x^b$ which is equivalent to shifting the digits left $b$ places and inserting zeroes
in the least significant digits.  Similarly to divide by a power of $x$ the following function is provided.

\index{mp\_rshd}
\begin{alltt}
void mp_rshd (mp_int *a, int b)
\end{alltt}
This will divide $a$ in place by $x^b$ and discard the remainder.  This function cannot fail as it performs the operations
in place and no new digits are required to complete it.

\subsection{AND, OR, XOR and COMPLEMENT Operations}

While AND, OR and XOR operations compute arbitrary-precision bitwise operations. Negative numbers
are treated as if they are in two-complement representation, while internally they are sign-magnitude however.

\index{mp\_or} \index{mp\_and} \index{mp\_xor} \index{mp\_complement}
\begin{alltt}
mp_err mp_or  (const mp_int *a, mp_int *b, mp_int *c);
mp_err mp_and (const mp_int *a, mp_int *b, mp_int *c);
mp_err mp_xor (const mp_int *a, mp_int *b, mp_int *c);
mp_err mp_complement(const mp_int *a, mp_int *b);
mp_err mp_signed_rsh(const mp_int *a, int b, mp_int *c, mp_int *d);
\end{alltt}

The function \texttt{mp\_complement} computes a two-complement $b = \sim a$. The function \texttt{mp\_signed\_rsh} performs
sign extending right shift. For positive numbers it is equivalent to \texttt{mp\_div\_2d}.

\section{Addition and Subtraction}

To compute an addition or subtraction the following two functions can be used.

\index{mp\_add} \index{mp\_sub}
\begin{alltt}
mp_err mp_add (const mp_int *a, const mp_int *b, mp_int *c);
mp_err mp_sub (const mp_int *a, const mp_int *b, mp_int *c)
\end{alltt}

Which perform $c = a \odot b$ where $\odot$ is one of signed addition or subtraction.  The operations are fully sign
aware.

\section{Sign Manipulation}
\subsection{Negation}
\label{sec:NEG}
Simple integer negation can be performed with the following.

\index{mp\_neg}
\begin{alltt}
mp_err mp_neg (const mp_int *a, mp_int *b);
\end{alltt}

Which assigns $-a$ to $b$.

\subsection{Absolute}
Simple integer absolutes can be performed with the following.

\index{mp\_abs}
\begin{alltt}
mp_err mp_abs (const mp_int *a, mp_int *b);
\end{alltt}

Which assigns $\vert a \vert$ to $b$.

\section{Integer Division and Remainder}
To perform a complete and general integer division with remainder use the following function.

\index{mp\_div}
\begin{alltt}
mp_err mp_div (const mp_int *a, const mp_int *b, mp_int *c, mp_int *d);
\end{alltt}

This divides $a$ by $b$ and stores the quotient in $c$ and $d$.  The signed quotient is computed such that
$bc + d = a$.  Note that either of $c$ or $d$ can be set to \texttt{NULL} if their value is not required.  If
$b$ is zero the function returns \texttt{MP\_VAL}.


\chapter{Multiplication and Squaring}
\section{Multiplication}
A full signed integer multiplication can be performed with the following.
\index{mp\_mul}
\begin{alltt}
mp_err mp_mul (const mp_int *a, const mp_int *b, mp_int *c);
\end{alltt}
Which assigns the full signed product $ab$ to $c$.  This function actually breaks into one of four cases which are
specific multiplication routines optimized for given parameters.  First there are the Toom-Cook multiplications which
should only be used with very large inputs.  This is followed by the Karatsuba multiplications which are for moderate
sized inputs.  Then followed by the Comba and baseline multipliers.

Fortunately for the developer you don't really need to know this unless you really want to fine tune the system.  mp\_mul()
will determine on its own\footnote{Some tweaking may be required but \texttt{make tune} will put some reasonable values in \texttt{bncore.c}} what routine to use automatically when it is called.

\begin{alltt}
int main(void)
\{
   mp_int number1, number2;
   mp_err result;

   /* Initialize the numbers */
   if ((result = mp_init_multi(&number1,
                               &number2, NULL)) != MP_OKAY) \{
      printf("Error initializing the numbers.  \%s",
             mp_error_to_string(result));
      return EXIT_FAILURE;
   \}

   /* set the terms */
   mp_set_i32(&number, 257);
   mp_set_i32(&number2, 1023);

   /* multiply them */
   if ((result = mp_mul(&number1, &number2,
                        &number1)) != MP_OKAY) \{
      printf("Error multiplying terms.  \%s",
             mp_error_to_string(result));
      return EXIT_FAILURE;
   \}

   /* display */
   printf("number1 * number2 == \%" PRIi32, mp_get_i32(&number1));

   /* free terms and return */
   mp_clear_multi(&number1, &number2, NULL);

   return EXIT_SUCCESS;
\}
\end{alltt}

If this program succeeds it shall output the following.

\begin{alltt}
number1 * number2 == 262911
\end{alltt}

\section{Squaring}
Since squaring can be performed faster than multiplication it is performed it's own function instead of just using
\texttt{mp\_mul}.

\index{mp\_sqr}
\begin{alltt}
mp_err mp_sqr (const mp_int *a, mp_int *b);
\end{alltt}

Will square $a$ and store it in $b$.  Like the case of multiplication there are four different squaring
algorithms all which can be called from the function \texttt{mp\_sqr}.  It is ideal to use \texttt{mp\_sqr} over \texttt{mp\_mul} when squaring terms because
of the speed difference.

\section{Tuning Polynomial Basis Routines}

Both of the Toom-Cook and Karatsuba multiplication algorithms are faster than the traditional $O(n^2)$ approach that
the Comba and baseline algorithms use.  At $O(n^{1.464973})$ and $O(n^{1.584962})$ running times respectively they require
considerably less work.  For example, a $10\,000$-digit multiplication would take roughly $724\,000$ single precision
multiplications with Toom-Cook or $100\,000\,000$ single precision multiplications with the standard Comba (a factor
of 138).

So why not always use Karatsuba or Toom-Cook?   The simple answer is that they have so much overhead that they're not
actually faster than Comba until you hit distinct  ``cutoff'' points.  For Karatsuba with the default configuration,
GCC 3.3.1 and an Athlon XP processor the cutoff point is roughly 110 digits (about 70 for the Intel P4).  That is, at
110 digits Karatsuba and Comba multiplications just about break even and for 110+ digits Karatsuba is faster.

To get reasonable values for the cut-off points for your architecture, type

\begin{alltt}
make tune
\end{alltt}

This will run a benchmark, computes the medians, rewrites \texttt{bncore.c}, and recompiles \texttt{bncore.c} and relinks the library.

The benchmark itself can be fine-tuned in the file \texttt{etc/tune\_it.sh}.

The program \texttt{etc/tune} is also able to print a list of values for printing curves with e.g.: \texttt{gnuplot}. type \texttt{./etc/tune -h} to get a list of all available options.

\chapter{Modular Reduction}

Modular reduction is process of taking the remainder of one quantity divided by another.  Expressed
as (\ref{eqn:mod}) the modular reduction is equivalent to the remainder of $b$ divided by $c$.

\begin{equation}
a \equiv b \mbox{ (mod }c\mbox{)}
\label{eqn:mod}
\end{equation}

Of particular interest to cryptography are reductions where $b$ is limited to the range $0 \le b < c^2$ since particularly
fast reduction algorithms can be written for the limited range.

Note that one of the four optimized reduction algorithms are automatically chosen in the modular exponentiation
algorithm \texttt{mp\_exptmod} when an appropriate modulus is detected.

\section{Straight Division}
In order to effect an arbitrary modular reduction the following algorithm is provided.

\index{mp\_mod}
\begin{alltt}
mp_err mp_mod(const mp_int *a,const  mp_int *b, mp_int *c);
\end{alltt}

This reduces $a$ modulo $b$ and stores the result in $c$.  The sign of $c$ shall agree with the sign
of $b$.  This algorithm accepts an input $a$ of any range and is not limited by $0 \le a < b^2$.

\section{Barrett Reduction}

Barrett reduction is a generic optimized reduction algorithm that requires pre--computation to achieve
a decent speedup over straight division.  First a $\mu$ value must be precomputed with the following function.

\index{mp\_reduce\_setup}
\begin{alltt}
mp_err mp_reduce_setup(const mp_int *a, mp_int *b);
\end{alltt}

Given a modulus in $b$ this produces the required $\mu$ value in $a$.  For any given modulus this only has to
be computed once.  Modular reduction can now be performed with the following.

\index{mp\_reduce}
\begin{alltt}
mp_err mp_reduce(const mp_int *a, const mp_int *b, mp_int *c);
\end{alltt}

This will reduce $a$ in place modulo $b$ with the precomputed $\mu$ value in $c$.  $a$ must be in the range
$0 \le a < b^2$.

\begin{alltt}
int main(void)
\{
   mp_int   a, b, c, mu;
   mp_err      result;

   /* initialize a,b to desired values, mp_init mu,
    * c and set c to 1...we want to compute a^3 mod b
    */

   /* get mu value */
   if ((result = mp_reduce_setup(&mu, b)) != MP_OKAY) \{
      printf("Error getting mu.  \%s",
             mp_error_to_string(result));
      return EXIT_FAILURE;
   \}

   /* square a to get c = a^2 */
   if ((result = mp_sqr(&a, &c)) != MP_OKAY) \{
      printf("Error squaring.  \%s",
             mp_error_to_string(result));
      return EXIT_FAILURE;
   \}

   /* now reduce `c' modulo b */
   if ((result = mp_reduce(&c, &b, &mu)) != MP_OKAY) \{
      printf("Error reducing.  \%s",
             mp_error_to_string(result));
      return EXIT_FAILURE;
   \}

   /* multiply a to get c = a^3 */
   if ((result = mp_mul(&a, &c, &c)) != MP_OKAY) \{
      printf("Error reducing.  \%s",
             mp_error_to_string(result));
      return EXIT_FAILURE;
   \}

   /* now reduce `c' modulo b  */
   if ((result = mp_reduce(&c, &b, &mu)) != MP_OKAY) \{
      printf("Error reducing.  \%s",
             mp_error_to_string(result));
      return EXIT_FAILURE;
   \}

   /* c now equals a^3 mod b */

   return EXIT_SUCCESS;
\}
\end{alltt}

This program will calculate $a^3 \mbox{ mod }b$ if all the functions succeed.

\section{Montgomery Reduction}

Montgomery is a specialized reduction algorithm for any odd moduli.  Like Barrett reduction a pre--computation
step is required.  This is accomplished with the following.

\index{mp\_montgomery\_setup}
\begin{alltt}
mp_err mp_montgomery_setup(const mp_int *a, mp_digit *mp);
\end{alltt}

For the given odd moduli $a$ the precomputation value is placed in $mp$.  The reduction is computed with the
following.

\index{mp\_montgomery\_reduce}
\begin{alltt}
mp_err mp_montgomery_reduce(mp_int *a, mp_int *m, mp_digit mp);
\end{alltt}
This reduces $a$ in place modulo $m$ with the pre--computed value $mp$.   $a$ must be in the range
$0 \le a < b^2$.

Montgomery reduction is faster than Barrett reduction for moduli smaller than the ``Comba'' limit.  With the default
setup for instance, the limit is $127$ digits ($3556$--bits).   Note that this function is not limited to
$127$ digits just that it falls back to a baseline algorithm after that point.

An important observation is that this reduction does not return $a \mbox{ mod }m$ but $aR^{-1} \mbox{ mod }m$
where $R = \beta^n$, $n$ is the n number of digits in $m$ and $\beta$ is the radix used (default is $2^{28}$).

To quickly calculate $R$ the following function was provided.

\index{mp\_montgomery\_calc\_normalization}
\begin{alltt}
mp_err mp_montgomery_calc_normalization(mp_int *a, mp_int *b);
\end{alltt}
Which calculates $a = R$ for the odd moduli $b$ without using multiplication or division.

The normal modus operandi for Montgomery reductions is to normalize the integers before entering the system.  For
example, to calculate $a^3 \mbox { mod }b$ using Montgomery reduction the value of $a$ can be normalized by
multiplying it by $R$.  Consider the following code snippet.

\begin{alltt}
int main(void)
\{
   mp_int   a, b, c, R;
   mp_digit mp;
   mp_err      result;

   /* initialize a,b to desired values,
    * mp_init R, c and set c to 1....
    */

   /* get normalization */
   if ((result = mp_montgomery_calc_normalization(&R, b)) != MP_OKAY) \{
      printf("Error getting norm.  \%s",
             mp_error_to_string(result));
      return EXIT_FAILURE;
   \}

   /* get mp value */
   if ((result = mp_montgomery_setup(&c, &mp)) != MP_OKAY) \{
      printf("Error setting up montgomery.  \%s",
             mp_error_to_string(result));
      return EXIT_FAILURE;
   \}

   /* normalize `a' so now a is equal to aR */
   if ((result = mp_mulmod(&a, &R, &b, &a)) != MP_OKAY) \{
      printf("Error computing aR.  \%s",
             mp_error_to_string(result));
      return EXIT_FAILURE;
   \}

   /* square a to get c = a^2R^2 */
   if ((result = mp_sqr(&a, &c)) != MP_OKAY) \{
      printf("Error squaring.  \%s",
             mp_error_to_string(result));
      return EXIT_FAILURE;
   \}

   /* now reduce `c' back down to c = a^2R^2 * R^-1 == a^2R */
   if ((result = mp_montgomery_reduce(&c, &b, mp)) != MP_OKAY) \{
      printf("Error reducing.  \%s",
             mp_error_to_string(result));
      return EXIT_FAILURE;
   \}

   /* multiply a to get c = a^3R^2 */
   if ((result = mp_mul(&a, &c, &c)) != MP_OKAY) \{
      printf("Error reducing.  \%s",
             mp_error_to_string(result));
      return EXIT_FAILURE;
   \}

   /* now reduce `c' back down to c = a^3R^2 * R^-1 == a^3R */
   if ((result = mp_montgomery_reduce(&c, &b, mp)) != MP_OKAY) \{
      printf("Error reducing.  \%s",
             mp_error_to_string(result));
      return EXIT_FAILURE;
   \}

   /* now reduce (again) `c' back down to c = a^3R * R^-1 == a^3 */
   if ((result = mp_montgomery_reduce(&c, &b, mp)) != MP_OKAY) \{
      printf("Error reducing.  \%s",
             mp_error_to_string(result));
      return EXIT_FAILURE;
   \}

   /* c now equals a^3 mod b */

   return EXIT_SUCCESS;
\}
\end{alltt}

This particular example does not look too efficient but it demonstrates the point of the algorithm.  By
normalizing the inputs the reduced results are always of the form $aR$ for some variable $a$.  This allows
a single final reduction to correct for the normalization and the fast reduction used within the algorithm.

For more details consider examining the file \texttt{bn\_mp\_exptmod\_fast.c}.

\section{Restricted Diminished Radix}

``Diminished Radix'' reduction refers to reduction with respect to moduli that are amenable to simple
digit shifting and small multiplications.  In this case the ``restricted'' variant refers to moduli of the
form $\beta^k - p$ for some $k \ge 0$ and $0 < p < \beta$ where $\beta$ is the radix (default to $2^{28}$).

As in the case of Montgomery reduction there is a pre--computation phase required for a given modulus.

\index{mp\_dr\_setup}
\begin{alltt}
void mp_dr_setup(const mp_int *a, mp_digit *d);
\end{alltt}

This computes the value required for the modulus $a$ and stores it in $d$.  This function cannot fail
and does not return any error codes.

To determine if $a$ is a valid DR modulus:
\index{mp\_dr\_is\_modulus}
\begin{alltt}
mp_bool mp_dr_is_modulus(const mp_int *a);
\end{alltt}

After the pre--computation a reduction can be performed with the following.

\index{mp\_dr\_reduce}
\begin{alltt}
mp_err mp_dr_reduce(mp_int *a, const mp_int *b, mp_digit mp);
\end{alltt}

This reduces $a$ in place modulo $b$ with the pre--computed value $mp$.  $b$ must be of a restricted
diminished radix form and $a$ must be in the range $0 \le a < b^2$.  Diminished radix reductions are
much faster than both Barrett and Montgomery reductions as they have a much lower asymptotic running time.

Since the moduli are restricted this algorithm is not particularly useful for something like Rabin, RSA or
BBS cryptographic purposes.  This reduction algorithm is useful for Diffie-Hellman and ECC where fixed
primes are acceptable.

Note that unlike Montgomery reduction there is no normalization process.  The result of this function is
equal to the correct residue.

\section{Unrestricted Diminished Radix}

Unrestricted reductions work much like the restricted counterparts except in this case the moduli is of the
form $2^k - p$ for $0 < p < \beta$.  In this sense the unrestricted reductions are more flexible as they
can be applied to a wider range of numbers.

\index{mp\_reduce\_2k\_setup}\index{mp\_reduce\_2k\_setup\_l}
\begin{alltt}
mp_err mp_reduce_2k_setup(const mp_int *a, mp_digit *d);
mp_err mp_reduce_2k_setup_l(const mp_int *a, mp_int *d);
\end{alltt}

This will compute the required $d$ value for the given moduli $a$.

\index{mp\_reduce\_2k}\index{mp\_reduce\_2k\_l}
\begin{alltt}
mp_err mp_reduce_2k(mp_int *a, const mp_int *n, mp_digit d);
mp_err mp_reduce_2k_l(mp_int *a, const mp_int *n, const mp_int *d);
\end{alltt}

This will reduce $a$ in place modulo $n$ with the pre--computed value $d$.  From my experience this routine is
slower than the function \texttt{mp\_dr\_reduce} but faster for most moduli sizes than the Montgomery reduction.

To determine if \texttt{mp\_reduce\_2k} can be used at all, ask the function \texttt{mp\_reduce\_is\_2k}.

\index{mp\_reduce\_is\_2k}\index{mp\_reduce\_is\_2k\_l}
\begin{alltt}
mp_bool mp_reduce_is_2k(const mp_int *a);
mp_bool mp_reduce_is_2k_l(const mp_int *a);
\end{alltt}

\section{Combined Modular Reduction}

Some of the combinations of an arithmetic operations followed by a modular reduction can be done in a faster way. The ones implemented are:

Addition $d = (a + b) \mod c$
\index{mp\_addmod}
\begin{alltt}
mp_err mp_addmod(const mp_int *a, const mp_int *b, const mp_int *c, mp_int *d);
\end{alltt}

Subtraction  $d = (a - b) \mod c$
\begin{alltt}
mp_err mp_submod(const mp_int *a, const mp_int *b, const mp_int *c, mp_int *d);
\end{alltt}

Multiplication $d = (ab) \mod c$
\begin{alltt}
mp_err mp_mulmod(const mp_int *a, const mp_int *b, const mp_int *c, mp_int *d);
\end{alltt}

Squaring  $d = (a^2) \mod c$
\begin{alltt}
mp_err mp_sqrmod(const mp_int *a, const mp_int *b, const mp_int *c, mp_int *d);
\end{alltt}



\chapter{Exponentiation}
\section{Single Digit Exponentiation}
\index{mp\_expt\_u32}
\begin{alltt}
mp_err mp_expt_u32 (const mp_int *a, uint32_t b, mp_int *c)
\end{alltt}
This function computes $c = a^b$.

\section{Modular Exponentiation}
\index{mp\_exptmod}
\begin{alltt}
mp_err mp_exptmod (const mp_int *G, const mp_int *X, const mp_int *P, mp_int *Y)
\end{alltt}
This computes $Y \equiv G^X \mbox{ (mod }P\mbox{)}$ using a variable width sliding window algorithm.  This function
will automatically detect the fastest modular reduction technique to use during the operation.  For negative values of
$X$ the operation is performed as $Y \equiv (G^{-1} \mbox{ mod }P)^{\vert X \vert} \mbox{ (mod }P\mbox{)}$ provided that
$gcd(G, P) = 1$.

This function is actually a shell around the two internal exponentiation functions.  This routine will automatically
detect when Barrett, Montgomery, Restricted and Unrestricted Diminished Radix based exponentiation can be used.  Generally
moduli of the a ``restricted diminished radix'' form lead to the fastest modular exponentiations.  Followed by Montgomery
and the other two algorithms.

\section{Modulus a Power of Two}
\index{mp\_mod\_2d}
\begin{alltt}
mp_err mp_mod_2d(const mp_int *a, int b, mp_int *c)
\end{alltt}
It calculates $c = a \mod 2^b$.

\section{Root Finding}
\index{mp\_n\_root}
\begin{alltt}
mp_err mp_root_u32(const mp_int *a, uint32_t b, mp_int *c)
\end{alltt}
This computes $c = a^{1/b}$ such that $c^b \le a$ and $(c+1)^b > a$. Will return a positive root only for even roots and return
a root with the sign of the input for odd roots.  For example, performing $4^{1/2}$ will return $2$ whereas $(-8)^{1/3}$
will return $-2$.

This algorithm uses the ``Newton Approximation'' method and will converge on the correct root fairly quickly.

The square root  $c = a^{1/2}$ (with the same conditions $c^2 \le a$ and $(c+1)^2 > a$) is implemented with a faster algorithm.

\index{mp\_sqrt}
\begin{alltt}
mp_err mp_sqrt(const mp_int *arg, mp_int *ret)
\end{alltt}


\chapter{Logarithm}
\section{Integer Logarithm}
A logarithm function for positive integer input \texttt{a, base} computing  $\floor{\log_bx}$ such that $(\log_b x)^b \le x$.
\index{mp\_ilogb}
\begin{alltt}
mp_err mp_log_u32(const mp_int *a, uint32_t base, uint32_t *c)
\end{alltt}
\subsection{Example}
\begin{alltt}
#include <stdlib.h>
#include <stdio.h>
#include <errno.h>

#include <tommath.h>

int main(int argc, char **argv)
{
   mp_int x, output;
   uint32_t base;
   mp_err e;

   if (argc != 3) {
      fprintf(stderr,"Usage %s base x\textbackslash{}n", argv[0]);
      exit(EXIT_FAILURE);
   }
   if ((e = mp_init_multi(&x, &output, NULL)) != MP_OKAY) {
      fprintf(stderr,"mp_init failed: \textbackslash{}"%s\textbackslash{}"\textbackslash{}n",
                     mp_error_to_string(e));
              exit(EXIT_FAILURE);
   }
   errno = 0;
#ifdef MP_64BIT
   /* Check for overflow skipped  */
   base = (uint32_t)strtoull(argv[1], NULL, 10);
#else
   base = (uint32_t)strtoul(argv[1], NULL, 10);
#endif
   if (errno == ERANGE) {
      fprintf(stderr,"strtoul(l) failed: input out of range\textbackslash{}n");
      exit(EXIT_FAILURE);
   }
   if ((e = mp_read_radix(&x, argv[2], 10)) != MP_OKAY) {
      fprintf(stderr,"mp_read_radix failed: \textbackslash{}"%s\textbackslash{}"\textbackslash{}n",
                      mp_error_to_string(e));
      exit(EXIT_FAILURE);
   }
   if ((e = mp_log_u32(&x, base, &output)) != MP_OKAY) {
      fprintf(stderr,"mp_ilogb failed: \textbackslash{}"%s\textbackslash{}"\textbackslash{}n",
                      mp_error_to_string(e));
      exit(EXIT_FAILURE);
   }

   if ((e = mp_fwrite(&output, 10, stdout)) != MP_OKAY) {
      fprintf(stderr,"mp_fwrite failed: \textbackslash{}"%s\textbackslash{}"\textbackslash{}n",
                      mp_error_to_string(e));
      exit(EXIT_FAILURE);
   }
   putchar('\textbackslash{}n');

   mp_clear_multi(&x, &output, NULL);
   exit(EXIT_SUCCESS);
}
\end{alltt}

\chapter{Prime Numbers}

\section{Fermat Test}
\index{mp\_prime\_fermat}
\begin{alltt}
mp_err mp_prime_fermat (const mp_int *a, const mp_int *b, int *result)
\end{alltt}
Performs a Fermat primality test to the base $b$.  That is it computes $b^a \mbox{ mod }a$ and tests whether the value is
equal to $b$ or not.  If the values are equal then $a$ is probably prime and $result$ is set to one.  Otherwise $result$
is set to zero.

\section{Miller-Rabin Test}
\index{mp\_prime\_miller\_rabin}
\begin{alltt}
mp_err mp_prime_miller_rabin (const mp_int *a, const mp_int *b, int *result)
\end{alltt}
Performs a Miller-Rabin test to the base $b$ of $a$.  This test is much stronger than the Fermat test and is very hard to
fool (besides with Carmichael numbers).  If $a$ passes the test (therefore is probably prime) $result$ is set to one.
Otherwise $result$ is set to zero.

Note that it is suggested that you use the Miller-Rabin test instead of the Fermat test since all of the failures of
Miller-Rabin are a subset of the failures of the Fermat test.

\subsection{Required Number of Tests}
Generally to ensure a number is very likely to be prime you have to perform the Miller-Rabin with at least a half-dozen
or so unique bases.  However, it has been proven that the probability of failure goes down as the size of the input goes up.
This is why a simple function has been provided to help out.

\index{mp\_prime\_rabin\_miller\_trials}
\begin{alltt}
mp_err mp_prime_rabin_miller_trials(int size)
\end{alltt}
This returns the number of trials required for a low probability of failure for a given \texttt{size} expressed in bits.  This comes in handy specially since larger numbers are slower to test. For example, a 512-bit number would require 18 tests for a probability of $2^{-160}$ whereas a 1024-bit number would only require 12 tests for a probability of $2^{-192}$. The exact values as implemented are listed in table \ref{table:millerrabinrunsimpl}.

\begin{table}[h]
\begin{center}
\begin{tabular}{c c c}
\textbf{bits} & \textbf{Rounds} & \textbf{Error}\\
 80 & -1  &  Use deterministic algorithm for size <= 80 bits \\
 81 & 37  &  $2^{-96}$ \\
 96 & 32  & $2^{-96}$ \\
 128 & 40  & $2^{-112}$ \\
 160 & 35  & $2^{-112}$ \\
 256 & 27  & $2^{-128}$ \\
 384 & 16  & $2^{-128}$ \\
 512 & 18  & $2^{-160}$ \\
 768 & 11  & $2^{-160}$ \\
 896 & 10  & $2^{-160}$ \\
 1024 & 12  & $2^{-192}$ \\
 1536 & 8   & $2^{-192}$ \\
 2048 & 6   & $2^{-192}$ \\
 3072 & 4   & $2^{-192}$ \\
 4096 & 5   & $2^{-256}$ \\
 5120 & 4   & $2^{-256}$ \\
 6144 & 4   & $2^{-256}$ \\
 8192 & 3   & $2^{-256}$ \\
 9216 & 3   & $2^{-256}$ \\
 10240 & 2  & $2^{-256}$
\end{tabular}
\caption{ Number of Miller-Rabin rounds as implemented } \label{table:millerrabinrunsimpl}
\end{center}
\end{table}

A small table, broke in two for typographical reasons, with the number of rounds of Miller-Rabin tests is shown below. The numbers have been computed with a PARI/GP script listed in appendix \ref{app:numberofmrcomp}.

The first column is the number of bits $b$ in the prime $p = 2^b$, the numbers in the first row represent the
probability that the number that all of the Miller-Rabin tests deemed a pseudoprime is actually a composite. There is a deterministic test for numbers smaller than $2^{80}$.

\begin{table}[h]
\begin{center}
\begin{tabular}{c c c c c c c}
\textbf{bits} & $\mathbf{2^{-80}}$ & $\mathbf{2^{-96}}$ & $\mathbf{2^{-112}}$ & $\mathbf{2^{-128}}$ & $\mathbf{2^{-160}}$ & $\mathbf{2^{-192}}$ \\
80    & 31 & 39 & 47 & 55 & 71 & 87  \\
96    & 29 & 37 & 45 & 53 & 69 & 85  \\
128   & 24 & 32 & 40 & 48 & 64 & 80  \\
160   & 19 & 27 & 35 & 43 & 59 & 75  \\
192   & 15 & 21 & 29 & 37 & 53 & 69  \\
256   & 10 & 15 & 20 & 27 & 43 & 59  \\
384   & 7  & 9  & 12 & 16 & 25 & 38  \\
512   & 5  & 7  & 9  & 12 & 18 & 26  \\
768   & 4  & 5  & 6  & 8  & 11 & 16  \\
1024  & 3  & 4  & 5  & 6  & 9  & 12  \\
1536  & 2  & 3  & 3  & 4  & 6  & 8   \\
2048  & 2  & 2  & 3  & 3  & 4  & 6   \\
3072  & 1  & 2  & 2  & 2  & 3  & 4   \\
4096  & 1  & 1  & 2  & 2  & 2  & 3   \\
6144  & 1  & 1  & 1  & 1  & 2  & 2   \\
8192  & 1  & 1  & 1  & 1  & 2  & 2   \\
12288 & 1  & 1  & 1  & 1  & 1  & 1   \\
16384 & 1  & 1  & 1  & 1  & 1  & 1   \\
24576 & 1  & 1  & 1  & 1  & 1  & 1   \\
32768 & 1  & 1  & 1  & 1  & 1  & 1
\end{tabular}
\caption{ Number of Miller-Rabin rounds. Part I } \label{table:millerrabinrunsp1}
\end{center}
\end{table}
\newpage
\begin{table}[h]
\begin{center}
\begin{tabular}{c c c c c c c c}
\textbf{bits} &$\mathbf{2^{-224}}$ & $\mathbf{2^{-256}}$ & $\mathbf{2^{-288}}$ & $\mathbf{2^{-320}}$ & $\mathbf{2^{-352}}$ & $\mathbf{2^{-384}}$ & $\mathbf{2^{-416}}$\\
80    & 103 & 119 & 135 & 151 & 167 & 183 & 199 \\
96    & 101 & 117 & 133 & 149 & 165 & 181 & 197 \\
128   & 96  & 112 & 128 & 144 & 160 & 176 & 192 \\
160   & 91  & 107 & 123 & 139 & 155 & 171 & 187 \\
192   & 85  & 101 & 117 & 133 & 149 & 165 & 181 \\
256   & 75  & 91  & 107 & 123 & 139 & 155 & 171 \\
384   & 54  & 70  & 86  & 102 & 118 & 134 & 150 \\
512   & 36  & 49  & 65  & 81  & 97  & 113 & 129 \\
768   & 22  & 29  & 37  & 47  & 58  & 70  & 86  \\
1024  & 16  & 21  & 26  & 33  & 40  & 48  & 58  \\
1536  & 10  & 13  & 17  & 21  & 25  & 30  & 35  \\
2048  & 8   & 10  & 13  & 15  & 18  & 22  & 26  \\
3072  & 5   & 7   & 8	& 10  & 12  & 14  & 17  \\
4096  & 4   & 5   & 6	& 8   & 9   & 11  & 12  \\
6144  & 3   & 4   & 4	& 5   & 6   & 7   & 8	\\
8192  & 2   & 3   & 3	& 4   & 5   & 6   & 6	\\
12288 & 2   & 2   & 2	& 3   & 3   & 4   & 4	\\
16384 & 1   & 2   & 2	& 2   & 3   & 3   & 3	\\
24576 & 1   & 1   & 2	& 2   & 2   & 2   & 2	\\
32768 & 1   & 1   & 1	& 1   & 2   & 2   & 2
\end{tabular}
\caption{ Number of Miller-Rabin rounds. Part II } \label{table:millerrabinrunsp2}
\end{center}
\end{table}

Determining the probability needed to pick the right column is a bit harder. Fips 186.4, for example has $2^{-80}$ for $512$ bit large numbers, $2^{-112}$ for $1024$ bits, and $2^{128}$ for $1536$ bits. It can be seen in table \ref{table:millerrabinrunsp1} that those combinations follow the diagonal from $(512,2^{-80})$ downwards and to the right to gain a lower probabilty of getting a composite declared a pseudoprime for the same amount of work or less.

If this version of the library has the strong Lucas-Selfridge and/or the Frobenius-Underwood test implemented only one or two rounds of the Miller-Rabin test with a random base is necesssary for numbers larger than or equal to $1024$ bits.

This function is meant for RSA. The number of rounds for DSA is $\lceil -log_2(p)/2\rceil$ with $p$ the probability which is just the half of the absolute value of $p$ if given as a power of two. E.g.: with $p = 2^{-128}$, $\lceil -log_2(p)/2\rceil = 64$.

This function can be used to test a DSA prime directly if these rounds are followed by a Lucas test.

See also table C.1 in FIPS 186-4.

\section{Strong Lucas-Selfridge Test}
\index{mp\_prime\_strong\_lucas\_selfridge}
\begin{alltt}
mp_err mp_prime_strong_lucas_selfridge(const mp_int *a, mp_bool *result)
\end{alltt}
Performs a strong Lucas-Selfridge test. The strong Lucas-Selfridge test together with the Rabin-Miler test with bases $2$ and $3$ resemble the BPSW test. The single internal use is a compile-time option in \texttt{mp\_prime\_is\_prime} and can be excluded
from the Libtommath build if not needed.

\section{Frobenius (Underwood)  Test}
\index{mp\_prime\_frobenius\_underwood}
\begin{alltt}
mp_err mp_prime_frobenius_underwood(const mp_int *N, mp_bool *result)
\end{alltt}
Performs the variant of the Frobenius test as described by Paul Underwood. It can be included at build-time if the preprocessor macro \texttt{LTM\_USE\_FROBENIUS\_TEST} is defined and will be used instead of the Lucas-Selfridge test.

It returns \texttt{MP\_ITER} if the number of iterations is exhausted, assumes a composite as the input and sets \texttt{result} accordingly. This will reduce the set of available pseudoprimes by a very small amount: test with large datasets (more than $10^{10}$ numbers, both randomly chosen and sequences of odd numbers with a random start point) found only 31 (thirty-one) numbers with $a > 120$ and none at all with just an additional simple check for divisors $d < 2^8$.

\section{Primality Testing}
Testing if a number is a square can be done a bit faster than just by calculating the square root. It is used by the primality testing function described below.
\index{mp\_is\_square}
\begin{alltt}
mp_err mp_is_square(const mp_int *arg, mp_bool *ret);
\end{alltt}


\index{mp\_prime\_is\_prime}
\begin{alltt}
mp_err mp_prime_is_prime(const mp_int *a, int t, mp_bool *result)
\end{alltt}
This will perform a trial division followed by two rounds of Miller-Rabin with bases 2 and 3 and a Lucas-Selfridge test. The Frobenius-Underwood is available as a compile-time option with the preprocessor macro \texttt{LTM\_USE\_FROBENIUS\_TEST}. See file
\texttt{bn\_mp\_prime\_is\_prime.c} for the necessary details. It shall be noted that both functions are much slower than
the Miller-Rabin test and if speed is an essential issue, the macro \texttt{LTM\_USE\_ONLY\_MR} switches the Frobenius-Underwood test and the Lucas-Selfridge test off and their code will not even be compiled into the library.

If $t$ is set to a positive value $t$ additional rounds of the Miller-Rabin test with random bases will be performed to allow for Fips 186.4 (vid.~p.~126ff) compliance. The function \texttt{mp\_prime\_rabin\_miller\_trials} can be used to determine the number of rounds. It is vital that the function \texttt{mp\_rand} has a cryptographically strong random number generator available.

One Miller-Rabin tests with a random base will be run automatically, so by setting $t$ to a positive value this function will run $t + 1$ Miller-Rabin tests with random bases.

If  $t$ is set to a negative value the test will run the deterministic Miller-Rabin test for the primes up to $3\,317\,044\,064\,679\,887\,385\,961\,981$\footnote{The semiprime $1287836182261\cdot 2575672364521$ with both factors smaller than $2^64$. An alternative with all factors smaller than $2^32$ is $4290067842\cdot 262853\cdot 1206721\cdot 2134439 + 3$}. That limit has to be checked by the caller.

If $a$ passes all of the tests $result$ is set to \texttt{MP\_YES}, otherwise it is set to \texttt{MP\_NO}.

\section{Next Prime}
\index{mp\_prime\_next\_prime}
\begin{alltt}
mp_err mp_prime_next_prime(mp_int *a, int t, mp_bool bbs_style)
\end{alltt}
This finds the next prime after $a$ that passes the function \texttt{mp\_prime\_is\_prime} with $t$ tests but see the documentation for
\texttt{mp\_prime\_is\_prime} for details regarding the use of the argument $t$.  Set $bbs\_style$ to \texttt{MP\_YES} if you
want only the next prime congruent to $3 \mbox{ mod } 4$, otherwise set it to \texttt{MP\_NO} to find any next prime.

\section{Random Primes}
\index{mp\_prime\_rand}
\begin{alltt}
mp_err mp_prime_rand(mp_int *a, int t, int size, int flags);
\end{alltt}
This will generate a prime in $a$ using $t$ tests of the primality testing algorithms.
See the documentation for mp\_prime\_is\_prime for details regarding the use of the argument $t$.
The variable $size$ specifies the bit length of the prime desired.
The variable $flags$ specifies one of several options available
(see fig. \ref{fig:primeopts}) which can be OR'ed together.

The function mp\_prime\_rand() is suitable for generating primes which must be secret (as in the case of RSA) since there
is no skew on the least significant bits.

\begin{figure}[h]
\begin{center}
\begin{small}
\begin{tabular}{|r|l|}
\hline \textbf{Flag}         & \textbf{Meaning} \\
\hline MP\_PRIME\_BBS       & Make the prime congruent to $3$ modulo $4$ \\
\hline MP\_PRIME\_SAFE      & Make a prime $p$ such that $(p - 1)/2$ is also prime. \\
                             & This option implies MP\_PRIME\_BBS as well. \\
\hline MP\_PRIME\_2MSB\_OFF & Makes sure that the bit adjacent to the most significant bit \\
                             & Is forced to zero.  \\
\hline MP\_PRIME\_2MSB\_ON  & Makes sure that the bit adjacent to the most significant bit \\
                             & Is forced to one. \\
\hline
\end{tabular}
\end{small}
\end{center}
\caption{Primality Generation Options}
\label{fig:primeopts}
\end{figure}

\chapter{Random Number Generation}
\section{PRNG}
\index{mp\_rand\_digit}
\begin{alltt}
mp_err mp_rand_digit(mp_digit *r)
\end{alltt}
This function generates a random number in \texttt{r} of the size given in \texttt{r} (that is, the variable is used for in- and output) but not more than \texttt{MP\_DIGIT\_MAX} bits.

\index{mp\_rand}
\begin{alltt}
mp_err mp_rand(mp_int *a, int digits)
\end{alltt}
This function generates a random number of \texttt{digits} bits.

The random number generated with these two functions is cryptographically secure if the source of random numbers the operating systems offers is cryptographically secure. It will use \texttt{arc4random()} if the OS is a BSD flavor, Wincrypt on Windows, or \texttt{/dev/urandom} on all operating systems that have it.

If you have a custom random source you might find the function \texttt(mp\_rand\_source) useful.
\index{mp\_rand\_source}
\begin{alltt}
void mp_rand_source(mp_err(*source)(void *out, size_t size));
\end{alltt}


\chapter{Input and Output}
\section{ASCII Conversions}
\subsection{To ASCII}
\index{mp\_to\_radix}
\begin{alltt}
mp_err mp_to_radix (const mp_int *a, char *str, size_t maxlen, size_t *written, int radix);
\end{alltt}
This stores $a$ in \texttt{str} of maximum length \texttt{maxlen} as a base-\texttt{radix} string of ASCII chars and appends a \texttt{NUL} character to terminate the string.

Valid values of \texttt{radix} are in the range $[2, 64]$.

The exact number of characters in \texttt{str} plus the \texttt{NUL} will be put in \texttt{written} if that variable is not set to \texttt{NULL}.

If \texttt{str} is not big enough to hold $a$, \texttt{str} will be filled with the least-significant digits
of length \texttt{maxlen-1}, then \texttt{str} will be \texttt{NUL} terminated and the error \texttt{MP\_BUF} is returned.

Please be aware that this function cannot evaluate the actual size of the buffer, it relies on the correctness of \texttt{maxlen}!


\index{mp\_radix\_size}
\begin{alltt}
mp_err mp_radix_size (const mp_int *a, int radix, int *size)
\end{alltt}
This stores in \texttt{size} the number of characters (including space for the NUL terminator) required.  Upon error this
function returns an error code and \texttt{size} will be zero.

If \texttt{MP\_NO\_FILE} is not defined a function to write to a file is also available.
\index{mp\_fwrite}
\begin{alltt}
mp_err mp_fwrite(const mp_int *a, int radix, FILE *stream);
\end{alltt}


\subsection{From ASCII}
\index{mp\_read\_radix}
\begin{alltt}
mp_err mp_read_radix (mp_int *a, const char *str, int radix);
\end{alltt}
This will read a \texttt{NUL} terminated string in base \texttt{radix} from \texttt{str} into $a$.  It will stop reading when it reads a
character it does not recognize (which happens to include the \texttt{NUL} char... imagine that...).  A single leading $-$ sign
can be used to denote a negative number.
The input encoding is currently restricted to ASCII only.

If \texttt{MP\_NO\_FILE} is not defined a function to read from a file is also available.
\index{mp\_fread}
\begin{alltt}
mp_err mp_fread(mp_int *a, int radix, FILE *stream);
\end{alltt}


\section{Binary Conversions}

Converting an \texttt{mp\_int} to and from binary is another keen idea.

\index{mp\_ubin\_size}
\begin{alltt}
size_t mp_ubin_size(const mp_int *a);
\end{alltt}

This will return the number of bytes (octets) required to store the unsigned copy of the integer $a$.

\index{mp\_to\_ubin}
\begin{alltt}
mp_err mp_to_ubin(const mp_int *a, unsigned char *buf, size_t maxlen, size_t *written)
\end{alltt}
This will store $a$ into the buffer $b$ of size \texttt{maxlen} in big--endian format storing the number of bytes written in \texttt{len}.  Fortunately this is exactly what DER (or is it ASN?) requires.  It does not store the sign of the integer.

\index{mp\_from\_ubin}
\begin{alltt}
mp_err mp_from_ubin(mp_int *a, unsigned char *b, size_t size);
\end{alltt}
This will read in an unsigned big--endian array of bytes (octets) from $b$ of length \texttt{size} into $a$.  The resulting big-integer $a$ will always be positive.

For those who acknowledge the existence of negative numbers (heretic!) there are ``signed'' versions of the
previous functions.
\index{mp\_signed\_bin\_size} \index{mp\_to\_signed\_bin} \index{mp\_read\_signed\_bin}
\begin{alltt}
size_t mp_sbin_size(const mp_int *a);
mp_err mp_from_sbin(mp_int *a, const unsigned char *b, size_t size);
mp_err mp_to_sbin(const mp_int *a, unsigned char *b, size_t maxsize, size_t *len);
\end{alltt}
They operate essentially the same as the unsigned copies except they prefix the data with zero or non--zero
byte depending on the sign.  If the sign is \texttt{MP\_ZPOS} (e.g. not negative) the prefix is zero, otherwise the prefix
is non--zero.

The two functions \texttt{mp\_unpack} (get your gifts out of the box, import binary data) and \texttt{mp\_pack} (put your gifts into the box, export binary data) implement the similarly working GMP functions as described at \url{http://gmplib.org/manual/Integer-Import-and-Export.html} with the exception that \texttt{mp\_pack} will not allocate memory if \texttt{rop} is \texttt{NULL}.
\index{mp\_unpack} \index{mp\_pack}
\begin{alltt}
mp_err mp_unpack(mp_int *rop, size_t count, mp_order order, size_t size,
             mp_endian endian, size_t nails, const void *op, size_t maxsize);
mp_err mp_pack(void *rop, size_t *countp, mp_order order, size_t size,
             mp_endian endian, size_t nails, const mp_int *op);
\end{alltt}
The function \texttt{mp\_pack} has the additional variable \texttt{maxsize} which must hold the size of the buffer \texttt{rop} in bytes. Use
\begin{alltt}
/* Parameters "nails" and "size" are the same as in mp_pack */
size_t mp_pack_count(const mp_int *a, size_t nails, size_t size);
\end{alltt}
To get the size in bytes necessary to be put in \texttt{maxsize}).

To enhance the readability of your code, the following enums have been wrought for your convenience.
\begin{alltt}
typedef enum {
   MP_LSB_FIRST = -1,
   MP_MSB_FIRST =  1
} mp_order;
typedef enum {
   MP_LITTLE_ENDIAN  = -1,
   MP_NATIVE_ENDIAN  =  0,
   MP_BIG_ENDIAN     =  1
} mp_endian;
\end{alltt}

\chapter{Algebraic Functions}
\section{Extended Euclidean Algorithm}
\index{mp\_exteuclid}
\begin{alltt}
mp_err mp_exteuclid(const mp_int *a, const mp_int *b,
                 mp_int *U1, mp_int *U2, mp_int *U3);
\end{alltt}

This finds the triple $U_1$/$U_2$/$U_3$ using the Extended Euclidean algorithm such that the following equation holds.

\begin{equation}
a \cdot U_1 + b \cdot U_2 = U_3
\end{equation}

Any of the \texttt{U1}/\texttt{U2}/\texttt{U3} parameters can be set to \textbf{NULL} if they are not desired.

\section{Greatest Common Divisor}
\index{mp\_gcd}
\begin{alltt}
mp_err mp_gcd (const mp_int *a, const mp_int *b, mp_int *c)
\end{alltt}
This will compute the greatest common divisor of $a$ and $b$ and store it in $c$.

\section{Least Common Multiple}
\index{mp\_lcm}
\begin{alltt}
mp_err mp_lcm (const mp_int *a, const mp_int *b, mp_int *c)
\end{alltt}
This will compute the least common multiple of $a$ and $b$ and store it in $c$.


\section{Kronecker Symbol}
\index{mp\_kronecker}
\begin{alltt}
mp_err mp_kronecker (const mp_int *a, const mp_int *p, int *c)
\end{alltt}
This will compute the Kronecker symbol (an extension of the Jacobi symbol) for $a$ with respect to $p$ with $\lbrace a, p \rbrace \in \mathbb{Z}$.  If $p$ is prime this essentially computes the Legendre
symbol.  The result is stored in $c$ and can take on one of three values $\lbrace -1, 0, 1 \rbrace$.  If $p$ is prime
then the result will be $-1$ when $a$ is not a quadratic residue modulo $p$.  The result will be $0$ if $a$ divides $p$
and the result will be $1$ if $a$ is a quadratic residue modulo $p$.


\section{Modular square root}
\index{mp\_sqrtmod\_prime}
\begin{alltt}
mp_err mp_sqrtmod_prime(const mp_int *n, const mp_int *p, mp_int *r)
\end{alltt}

This will solve the modular equation $r^2 = n \mod p$ where $p$ is a prime number greater than 2 (odd prime).
The result is returned in the third argument $r$, the function returns \texttt{MP\_OKAY} on success,
other return values indicate failure.

The implementation is split for two different cases:

1. if $p \mod 4 == 3$ we apply \href{http://cacr.uwaterloo.ca/hac/}{Handbook of Applied Cryptography algorithm 3.36} and compute $r$ directly as
$r = n^{(p+1)/4} \mod p$

2. otherwise we use \href{https://en.wikipedia.org/wiki/Tonelli-Shanks_algorithm}{Tonelli-Shanks algorithm}

The function does not check the primality of parameter $p$ thus it is up to the caller to assure that this parameter
is a prime number. When $p$ is a composite the function behaviour is undefined, it may even return a false-positive
\texttt{MP\_OKAY}.

\section{Modular Inverse}
\index{mp\_invmod}
\begin{alltt}
mp_err mp_invmod (const mp_int *a, const mp_int *b, mp_int *c)
\end{alltt}
Computes the multiplicative inverse of $a$ modulo $b$ and stores the result in $c$ such that $ac \equiv 1 \mbox{ (mod }b\mbox{)}$.

\section{Single Digit Functions}

For those using small numbers (\textit{snicker snicker}) there are several ``helper'' functions

\index{mp\_add\_d} \index{mp\_sub\_d} \index{mp\_mul\_d} \index{mp\_div\_d} \index{mp\_mod\_d}
\begin{alltt}
mp_err mp_add_d(const mp_int *a, mp_digit b, mp_int *c);
mp_err mp_sub_d(const mp_int *a, mp_digit b, mp_int *c);
mp_err mp_mul_d(const mp_int *a, mp_digit b, mp_int *c);
mp_err mp_div_d(const mp_int *a, mp_digit b, mp_int *c, mp_digit *d);
mp_err mp_mod_d(const mp_int *a, mp_digit b, mp_digit *c);
\end{alltt}

These work like the full \texttt{mp\_int} capable variants except the second parameter $b$ is a \texttt{mp\_digit}.  These
functions fairly handy if you have to work with relatively small numbers since you will not have to allocate
an entire \texttt{mp\_int} to store a number like $1$ or $2$.

The functions \texttt{mp\_incr} and \texttt{mp\_decr} mimic the postfix operators \texttt{++} and \texttt{--} respectively, to increment the input by one. They call the full single-digit functions if the addition would carry. Both functions need to be included in a minimized library because they call each other in case of a negative input, These functions change the inputs!
\begin{alltt}
mp_err mp_incr(mp_int *a);
mp_err mp_decr(mp_int *a);
\end{alltt}


The division by three can be made faster by replacing the division with a multiplication by the multiplicative inverse of three.

\index{mp\_div\_3}
\begin{alltt}
mp_err mp_div_3(const mp_int *a, mp_int *c, mp_digit *d);
\end{alltt}

\chapter{Little Helpers}
It is never wrong to have some useful little shortcuts at hand.
\section{Function Macros}
To make this overview simpler the macros are given as function prototypes. The return of logic macros is \texttt{MP\_NO} or \texttt{MP\_YES} respectively.

\index{mp\_iseven}
\begin{alltt}
mp_bool mp_iseven(const mp_int *a)
\end{alltt}
Checks if $a = 0 mod 2$

\index{mp\_isodd}
\begin{alltt}
mp_bool mp_isodd(const mp_int *a)
\end{alltt}
Checks if $a = 1 mod 2$

\index{mp\_isneg}
\begin{alltt}
mp_bool mp_isneg(mp_int *a)
\end{alltt}
Checks if $a < 0$


\index{mp\_iszero}
\begin{alltt}
mp_bool mp_iszero(mp_int *a)
\end{alltt}
Checks if $a = 0$. It does not check if the amount of memory allocated for $a$ is also minimal.


Other macros which are either shortcuts to normal functions or just other names for them do have their place in a programmer's life, too!

\subsection{Renamings}
\index{mp\_mag\_size}
\begin{alltt}
#define mp_mag_size(mp) mp_unsigned_bin_size(mp)
\end{alltt}


\index{mp\_raw\_size}
\begin{alltt}
#define mp_raw_size(mp) mp_signed_bin_size(mp)
\end{alltt}


\index{mp\_read\_mag}
\begin{alltt}
#define mp_read_mag(mp, str, len) mp_read_unsigned_bin((mp), (str), (len))
\end{alltt}


\index{mp\_read\_raw}
\begin{alltt}
 #define mp_read_raw(mp, str, len) mp_read_signed_bin((mp), (str), (len))
\end{alltt}


\index{mp\_tomag}
\begin{alltt}
#define mp_tomag(mp, str) mp_to_unsigned_bin((mp), (str))
\end{alltt}


\index{mp\_toraw}
\begin{alltt}
#define mp_toraw(mp, str)         mp_to_signed_bin((mp), (str))
\end{alltt}



\subsection{Shortcuts}

\index{mp\_to\_binary}
\begin{alltt}
#define mp_to_binary(M, S, N)  mp_to_radix((M), (S), (N), 2)
\end{alltt}


\index{mp\_to\_octal}
\begin{alltt}
#define mp_to_octal(M, S, N)   mp_to_radix((M), (S), (N), 8)
\end{alltt}


\index{mp\_to\_decimal}
\begin{alltt}
#define mp_to_decimal(M, S, N) mp_to_radix((M), (S), (N), 10)
\end{alltt}


\index{mp\_to\_hex}
\begin{alltt}
#define mp_to_hex(M, S, N)     mp_to_radix((M), (S), (N), 16)
\end{alltt}

\begin{appendices}
\appendixpage
%\noappendicestocpagenum
\addappheadtotoc
\chapter{Computing Number of Miller-Rabin Trials}\label{app:numberofmrcomp}
The number of Miller-Rabin rounds in the tables \ref{millerrabinrunsimpl}, \ref{millerrabinrunsp1}, and \ref{millerrabinrunsp2} have been calculated with the formula in FIPS 186-4 appendix F.1 (page 117) implemented as a PARI/GP script.
\begin{alltt}
log2(x) = log(x)/log(2)

fips_f1_sums(k, M, t) = {
   local(s = 0);
   s = sum(m=3,M,
          2^(m-t*(m-1)) *
          sum(j=2,m,
             1/ ( 2^( j + (k-1)/j ) )
          )
        );
   return(s);
}

fips_f1_2(k, t, M) = {
   local(common_factor, t1, t2, f1, f2, ds, res);

   common_factor = 2.00743 * log(2) * k * 2^(-k);
   t1 = 2^(k - 2 - M*t);
   f1 = (8 * ((Pi^2) - 6))/3;
   f2 = 2^(k - 2);
   ds = t1 + f1 * f2 * fips_f1_sums(k, M, t);
   res = common_factor * ds;
   return(res);
}

fips_f1_1(prime_length, ptarget)={
   local(t, t_end, M, M_end, pkt);

   t_end = ceil(-log2(ptarget)/2);
   M_end = floor(2 * sqrt(prime_length-1) - 1);

   for(t = 1, t_end,
      for(M = 3, M_end,
         pkt = fips_f1_2(prime_length, t, M);
         if(pkt <= ptarget,
            return(t);
         );
      );
   );
}
\end{alltt}

To get the number of rounds for a $1024$ bit large prime with a probability of $2^{-160}$:
\begin{alltt}
? fips_f1_1(1024,2^(-160))
%1 = 9
\end{alltt}
\end{appendices}
\documentclass[]{article}
\begin{document}

\title{LibTomMath v0.24 \\ A Free Multiple Precision Integer Library \\ http://math.libtomcrypt.org }
\author{Tom St Denis \\ tomstdenis@iahu.ca}
\maketitle
\newpage

\section{Introduction}
``LibTomMath'' is a free and open source library that provides multiple-precision integer functions required to form a 
basis of a public key cryptosystem.  LibTomMath is written entire in portable ISO C source code and designed to have an 
application interface much like that of MPI from Michael Fromberger.  

LibTomMath was written from scratch by Tom St Denis but designed to be  drop in replacement for the MPI package.  The 
algorithms within the library are derived from descriptions as provided in the Handbook of Applied Cryptography and Knuth's
``The Art of Computer Programming''.  The library has been extensively optimized and should provide quite comparable 
timings as compared to many free and commercial libraries.

LibTomMath was designed with the following goals in mind:
\begin{enumerate}
\item Be a drop in replacement for MPI.
\item Be much faster than MPI.
\item Be written entirely in portable C.
\end{enumerate}

All three goals have been achieved to one extent or another (actual figures depend on what platform you are using).

Being compatible with MPI means that applications that already use it can be ported fairly quickly.  Currently there are 
a few differences but there are many similarities.  In fact the average MPI based application can be ported in under 15
minutes.  

Thanks goes to Michael Fromberger for answering a couple questions and Colin Percival for having the patience and courtesy to
help debug and suggest optimizations.  They were both of great help!

\section{Building Against LibTomMath}

As of v0.12 LibTomMath is not a simple single source file project like MPI.  LibTomMath retains the exact same API as MPI
but is implemented differently.  To build LibTomMath you will need a copy of GNU cc and GNU make.  Both are free so if you 
don't have a copy don't whine to me about it.

To build the library type 

\begin{verbatim}
make
\end{verbatim}

This will build the library file libtommath.a.  If you want to build the library and also install it (in /usr/bin and /usr/include) then
type 

\begin{verbatim}
make install
\end{verbatim}

Now within your application include ``tommath.h'' and link against libtommath.a to get MPI-like functionality.

\subsection{Microsoft Visual C++}
A makefile is also provided for MSVC (\textit{tested against MSVC 6.00 with SP5}) which allows the library to be used
with that compiler as well.  To build the library type

\begin{verbatim}
nmake -f makefile.msvc
\end{verbatim}

Which will build ``tommath.lib''.  

\section{Programming with LibTomMath}

\subsection{The mp\_int Structure}
All multiple precision integers are stored in a structure called \textbf{mp\_int}.  A multiple precision integer is
essentially an array of \textbf{mp\_digit}.  mp\_digit is defined at the top of ``tommath.h''.  The type can be changed 
to suit a particular platform.  

For example, when \textbf{MP\_8BIT} is defined a mp\_digit is a unsigned char and holds seven bits.  Similarly 
when \textbf{MP\_16BIT} is defined a mp\_digit is a unsigned short and holds 15 bits.   By default a mp\_digit is a 
unsigned long and holds 28 bits which is optimal for most 32 and 64 bit processors.

The choice of digit is particular to the platform at hand and what available multipliers are provided.  For 
MP\_8BIT either a $8 \times 8 \Rightarrow 16$ or $16 \times 16 \Rightarrow 16$ multiplier is optimal.  When 
MP\_16BIT is defined either a $16 \times 16 \Rightarrow 32$ or $32 \times 32 \Rightarrow 32$ multiplier is optimal.  By
default a $32 \times 32 \Rightarrow 64$ or $64 \times 64 \Rightarrow 64$ multiplier is optimal.  

This gives the library some flexibility.  For example, a i8051 has a $8 \times 8 \Rightarrow 16$ multiplier.  The 
16-bit x86 instruction set has a $16 \times 16 \Rightarrow 32$ multiplier.  In practice this library is not particularly
designed for small devices like an i8051 due to the size.  It is possible to strip out functions which are not required 
to drop the code size.  More realistically the library is well suited to 32 and 64-bit processors that have decent
integer multipliers.  The AMD Athlon XP and Intel Pentium 4 processors are examples of well suited processors.

Throughout the discussions there will be references to a \textbf{used} and \textbf{alloc} members of an integer.  The
used member refers to how many digits are actually used in the representation of the integer.  The alloc member refers
to how many digits have been allocated off the heap.  There is also the $\beta$ quantity which is equal to $2^W$ where 
$W$ is the number of bits in a digit (default is 28).  

\subsection{Calling Functions}
Most functions expect pointers to mp\_int's as parameters.   To save on memory usage it is possible to have source
variables as destinations.  The arguements are read left to right so to compute $x + y = z$ you would pass the arguments
in the order $x, y, z$.  For example:
\begin{verbatim}
   mp_add(&x, &y, &x);           /* x = x + y */
   mp_mul(&y, &x, &z);           /* z = y * x */
   mp_div_2(&x, &y);             /* y = x / 2 */
\end{verbatim}

\subsection{Various Optimizations}
Various routines come in several ``flavours'' which are optimized for particular cases of inputs.  For instance
the multiplicative inverse function ``mp\_invmod()'' has a routine for odd and even moduli.  Similarly the
``mp\_exptmod()'' function has several variants depending on the modulus as well.  Several lower level
functions such as multiplication, squaring and reductions come in ``comba'' and ``baseline'' variants.

The design of LibTomMath is such that the end user does not have to concern themselves too much with these
details.  This is why the functions provided will determine \textit{automatically} when an appropriate
optimal function can be used.  For example, when you call ``mp\_mul()'' the routines will first determine
if the Karatsuba multiplier should be used.  If not it will determine if the ``comba'' method can be used
and finally call the standard catch-all ``baseline'' method.

Throughout the rest of this manual several variants for various functions will be referenced to as
the ``comba'', ``baseline'', etc... method.  Keep in mind you call one function to use any of the optimal
variants.

\subsection{Return Values}
All functions that return errors will return \textbf{MP\_OKAY} if the function was succesful.  It will return 
\textbf{MP\_MEM} if it ran out of heap memory or \textbf{MP\_VAL} if one of the arguements is out of range.  

\subsection{Basic Functionality}
Before an mp\_int can be used it must be initialized with 

\begin{verbatim}
int mp_init(mp_int *a);
\end{verbatim}

For example, consider the following.

\begin{verbatim}
#include "tommath.h"
int main(void)
{
   mp_int num;
   if (mp_init(&num) != MP_OKAY) {
      printf("Error initializing a mp_int.\n");
   }
   return 0;
}   
\end{verbatim}

A mp\_int can be freed from memory with

\begin{verbatim}
void mp_clear(mp_int *a);
\end{verbatim}

This will zero the memory and free the allocated data.  There are a set of trivial functions to manipulate the 
value of an mp\_int.  

\begin{verbatim}
/* set to zero */
void mp_zero(mp_int *a);

/* set to a digit */
void mp_set(mp_int *a, mp_digit b);

/* set a 32-bit const */
int mp_set_int(mp_int *a, unsigned long b);

/* init to a given number of digits */
int mp_init_size(mp_int *a, int size);

/* copy, b = a */
int mp_copy(mp_int *a, mp_int *b);

/* inits and copies, a = b */
int mp_init_copy(mp_int *a, mp_int *b);
\end{verbatim}

The \textbf{mp\_zero} function will clear the contents of a mp\_int and set it to positive.  The \textbf{mp\_set} function 
will zero the integer and set the first digit to a value specified.  The \textbf{mp\_set\_int} function will zero the 
integer and set the first 32-bits to a given value.  It is important to note that using mp\_set can have unintended 
side effects when either the  MP\_8BIT or MP\_16BIT defines are enabled.  By default the library will accept the 
ranges of values MPI will (and more).

The \textbf{mp\_init\_size} function will initialize the integer and set the allocated size to a given value.  The 
allocated digits are zero'ed by default but not marked as used.  The \textbf{mp\_copy} function will copy the digits
(and sign) of the first parameter into the integer specified by the second parameter.  The \textbf{mp\_init\_copy} will
initialize the first integer specified and copy the second one into it.  Note that the order is reversed from that of
mp\_copy.  This odd ``bug'' was kept to maintain compatibility with MPI.

\subsection{Digit Manipulations}

There are a class of functions that provide simple digit manipulations such as shifting and modulo reduction of powers
of two.  

\begin{verbatim}
/* right shift by "b" digits */
void mp_rshd(mp_int *a, int b);

/* left shift by "b" digits */
int mp_lshd(mp_int *a, int b);

/* c = a / 2^b */
int mp_div_2d(mp_int *a, int b, mp_int *c);

/* b = a/2 */
int mp_div_2(mp_int *a, mp_int *b);

/* c = a * 2^b */
int mp_mul_2d(mp_int *a, int b, mp_int *c);

/* b = a*2 */
int mp_mul_2(mp_int *a, mp_int *b);

/* c = a mod 2^d */
int mp_mod_2d(mp_int *a, int b, mp_int *c);

/* computes a = 2^b */
int mp_2expt(mp_int *a, int b);

/* makes a pseudo-random int of a given size */
int mp_rand(mp_int *a, int digits);

\end{verbatim}

\subsection{Binary Operations}

\begin{verbatim}

/* c = a XOR b  */
int mp_xor(mp_int *a, mp_int *b, mp_int *c);

/* c = a OR b */
int mp_or(mp_int *a, mp_int *b, mp_int *c);

/* c = a AND b */
int mp_and(mp_int *a, mp_int *b, mp_int *c);

\end{verbatim}

\subsection{Basic Arithmetic}

Next are the class of functions which provide basic arithmetic.

\begin{verbatim}
/* b = -a */
int mp_neg(mp_int *a, mp_int *b);

/* b = |a| */
int mp_abs(mp_int *a, mp_int *b);

/* compare a to b */
int mp_cmp(mp_int *a, mp_int *b);

/* compare |a| to |b| */
int mp_cmp_mag(mp_int *a, mp_int *b);

/* c = a + b */
int mp_add(mp_int *a, mp_int *b, mp_int *c);

/* c = a - b */
int mp_sub(mp_int *a, mp_int *b, mp_int *c);

/* c = a * b */
int mp_mul(mp_int *a, mp_int *b, mp_int *c);

/* b = a^2 */
int mp_sqr(mp_int *a, mp_int *b);

/* a/b => cb + d == a */
int mp_div(mp_int *a, mp_int *b, mp_int *c, mp_int *d);

/* c = a mod b, 0 <= c < b  */
int mp_mod(mp_int *a, mp_int *b, mp_int *c);
\end{verbatim}

\subsection{Single Digit Functions}

\begin{verbatim}
/* compare against a single digit */
int mp_cmp_d(mp_int *a, mp_digit b);

/* c = a + b */
int mp_add_d(mp_int *a, mp_digit b, mp_int *c);

/* c = a - b */
int mp_sub_d(mp_int *a, mp_digit b, mp_int *c);

/* c = a * b */
int mp_mul_d(mp_int *a, mp_digit b, mp_int *c);

/* a/b => cb + d == a */
int mp_div_d(mp_int *a, mp_digit b, mp_int *c, mp_digit *d);

/* c = a^b */
int mp_expt_d(mp_int *a, mp_digit b, mp_int *c);

/* c = a mod b, 0 <= c < b  */
int mp_mod_d(mp_int *a, mp_digit b, mp_digit *c);
\end{verbatim}

Note that care should be taken for the value of the digit passed.  By default, any 28-bit integer is a valid digit that can
be passed into the function.  However, if MP\_8BIT or MP\_16BIT is defined only 7 or 15-bit (respectively) integers 
can be passed into it.

\subsection{Modular Arithmetic}

There are some trivial modular arithmetic functions.

\begin{verbatim}
/* d = a + b (mod c) */
int mp_addmod(mp_int *a, mp_int *b, mp_int *c, mp_int *d);

/* d = a - b (mod c) */
int mp_submod(mp_int *a, mp_int *b, mp_int *c, mp_int *d);

/* d = a * b (mod c) */
int mp_mulmod(mp_int *a, mp_int *b, mp_int *c, mp_int *d);

/* c = a * a (mod b) */
int mp_sqrmod(mp_int *a, mp_int *b, mp_int *c);

/* c = 1/a (mod b) */
int mp_invmod(mp_int *a, mp_int *b, mp_int *c);

/* c = (a, b) */
int mp_gcd(mp_int *a, mp_int *b, mp_int *c);

/* c = [a, b] or (a*b)/(a, b) */
int mp_lcm(mp_int *a, mp_int *b, mp_int *c);

/* find the b'th root of a  */
int mp_n_root(mp_int *a, mp_digit b, mp_int *c);

/* computes the jacobi c = (a | n) (or Legendre if b is prime)  */
int mp_jacobi(mp_int *a, mp_int *n, int *c);

/* used to setup the Barrett reduction for a given modulus b */
int mp_reduce_setup(mp_int *a, mp_int *b);

/* Barrett Reduction, computes a (mod b) with a precomputed value c  
 *
 * Assumes that 0 < a <= b^2, note if 0 > a > -(b^2) then you can merely
 * compute the reduction as -1 * mp_reduce(mp_abs(a)) [pseudo code].
 */
int mp_reduce(mp_int *a, mp_int *b, mp_int *c);

/* setups the montgomery reduction */
int mp_montgomery_setup(mp_int *a, mp_digit *mp);

/* computes xR^-1 == x (mod N) via Montgomery Reduction */
int mp_montgomery_reduce(mp_int *a, mp_int *m, mp_digit mp);

/* returns 1 if a is a valid DR modulus */
int mp_dr_is_modulus(mp_int *a);

/* sets the value of "d" required for mp_dr_reduce */
void mp_dr_setup(mp_int *a, mp_digit *d);

/* reduces a modulo b using the Diminished Radix method */
int mp_dr_reduce(mp_int *a, mp_int *b, mp_digit mp);

/* d = a^b (mod c) */
int mp_exptmod(mp_int *a, mp_int *b, mp_int *c, mp_int *d);
\end{verbatim}

\subsection{Primality Routines}
\begin{verbatim}
/* ---> Primes <--- */
/* table of first 256 primes */
extern const mp_digit __prime_tab[];

/* result=1 if a is divisible by one of the first 256 primes */
int mp_prime_is_divisible(mp_int *a, int *result);

/* performs one Fermat test of "a" using base "b".  
 * Sets result to 0 if composite or 1 if probable prime 
 */
int mp_prime_fermat(mp_int *a, mp_int *b, int *result);

/* performs one Miller-Rabin test of "a" using base "b".
 * Sets result to 0 if composite or 1 if probable prime 
 */
int mp_prime_miller_rabin(mp_int *a, mp_int *b, int *result);

/* performs t rounds of Miller-Rabin on "a" using the first
 * t prime bases.  Also performs an initial sieve of trial
 * division.  Determines if "a" is prime with probability
 * of error no more than (1/4)^t.
 *
 * Sets result to 1 if probably prime, 0 otherwise
 */
int mp_prime_is_prime(mp_int *a, int t, int *result);

/* finds the next prime after the number "a" using "t" trials
 * of Miller-Rabin.
 */
int mp_prime_next_prime(mp_int *a, int t, int bbs_style);
\end{verbatim}

\subsection{Radix Conversions}
To read or store integers in other formats there are the following functions.

\begin{verbatim}
int mp_unsigned_bin_size(mp_int *a);
int mp_read_unsigned_bin(mp_int *a, unsigned char *b, int c);
int mp_to_unsigned_bin(mp_int *a, unsigned char *b);

int mp_signed_bin_size(mp_int *a);
int mp_read_signed_bin(mp_int *a, unsigned char *b, int c);
int mp_to_signed_bin(mp_int *a, unsigned char *b);

int mp_read_radix(mp_int *a, unsigned char *str, int radix);
int mp_toradix(mp_int *a, unsigned char *str, int radix);
int mp_radix_size(mp_int *a, int radix);
\end{verbatim}

The integers are stored in big endian format as most libraries (and MPI) expect.  The \textbf{mp\_read\_radix} and 
\textbf{mp\_toradix} functions read and write (respectively) null terminated ASCII strings in a given radix.  Valid values
for the radix are between 2 and 64 (inclusively).  

\section{Function Analysis}

Throughout the function analysis the variable $N$ will denote the average size of an input to a function as measured 
by the number of digits it has.  The variable $W$ will denote the number of bits per word and $c$ will denote a small
constant amount of work.  The big-oh notation will be abused slightly to consider numbers that do not grow to infinity.
That is we shall consider $O(N/2) \ne O(N)$ which is an abuse of the notation.

\subsection{Digit Manipulation Functions}
The class of digit manipulation functions such as \textbf{mp\_rshd}, \textbf{mp\_lshd} and \textbf{mp\_mul\_2} are all
very simple functions to analyze.  

\subsubsection{mp\_rshd(mp\_int *a, int b)}
Shifts $a$ by given number of digits to the right and is equivalent to dividing by $\beta^b$.  The work is performed
in-place which means the input and output are the same.  If the shift count $b$ is less than or equal to zero 
the function returns without doing any work.  If the the shift count is larger than the number of digits in $a$ 
then $a$ is simply zeroed without shifting digits.

This function requires no additional memory and $O(N)$ time.

\subsubsection{mp\_lshd(mp\_int *a, int b)}
Shifts $a$ by a given number of digits to the left and is equivalent to multiplying by $\beta^b$.  The work
is performed in-place which means the input and output are the same.  If the shift count $b$ is less than or equal 
to zero the function returns success without doing any work.

This function requires $O(b)$ additional digits of memory and $O(N)$ time.

\subsubsection{mp\_div\_2d(mp\_int *a, int b, mp\_int *c, mp\_int *d)}
Shifts $a$ by a given number of \textbf{bits} to the right and is equivalent to dividing by $2^b$.  The shifted number is stored
in the $c$ parameter.  The remainder of $a/2^b$ is optionally stored in $d$ (if it is not passed as NULL).  
If the shift count $b$ is less than or equal to zero the function places $a$ in $c$ and returns success.  

This function requires $O(2 \cdot N)$ additional digits of memory and $O(2 \cdot N)$ time.

\subsubsection{mp\_mul\_2d(mp\_int *a, int b, mp\_int *c)}
Shifts $a$ by a given number of bits to the left and is equivalent to multiplying by $2^b$.  The shifted number
is placed in the $c$ parameter.  If the shift count $b$ is less than or equal to zero the function places $a$
in $c$ and returns success.  

This function requires $O(N)$ additional digits of memory and $O(2 \cdot N)$ time.

\subsubsection{mp\_mul\_2(mp\_int *a, mp\_int *b)}
Multiplies $a$ by two and stores in $b$.  This function is hard coded todo a shift by one place so it is faster
than calling mp\_mul\_2d with a count of one.  

This function requires $O(N)$ additional digits of memory and $O(N)$ time.

\subsubsection{mp\_div\_2(mp\_int *a, mp\_int *b)}
Divides $a$ by two and stores in $b$.  This function is hard coded todo a shift by one place so it is faster
than calling mp\_div\_2d with a count of one.

This function requires $O(N)$ additional digits of memory and $O(N)$ time.

\subsubsection{mp\_mod\_2d(mp\_int *a, int b, mp\_int *c)}
Performs the action of reducing $a$ modulo $2^b$ and stores the result in $c$.  If the shift count $b$ is less than 
or equal to zero the function places $a$ in $c$ and returns success.  

This function requires $O(N)$ additional digits of memory and $O(2 \cdot N)$ time.

\subsubsection{mp\_2expt(mp\_int *a, int b)}
Computes $a = 2^b$ by first setting $a$ to zero then OR'ing the correct bit to get the right value.

\subsubsection{mp\_rand(mp\_int *a, int digits)}
Computes a pseudo-random (\textit{via rand()}) integer that is always ``$digits$'' digits in length.  Not for
cryptographic use.

\subsection{Binary Arithmetic}
\subsubsection{mp\_xor(mp\_int *a, mp\_int *b, mp\_int *c)}
Computes $c = a \oplus b$, pseudo-extends with zeroes whichever of $a$ or $b$ is shorter such that the length
of $c$ is the maximum length of the two inputs.

\subsubsection{mp\_or(mp\_int *a, mp\_int *b, mp\_int *c)}
Computes $c = a \lor b$, pseudo-extends with zeroes whichever of $a$ or $b$ is shorter such that the length
of $c$ is the maximum length of the two inputs.

\subsubsection{mp\_and(mp\_int *a, mp\_int *b, mp\_int *c)}
Computes $c = a \land b$, pseudo-extends with zeroes whichever of $a$ or $b$ is shorter such that the length
of $c$ is the maximum length of the two inputs.

\subsection{Basic Arithmetic}

\subsubsection{mp\_cmp(mp\_int *a, mp\_int *b)}
Performs a \textbf{signed} comparison between $a$ and $b$ returning \textbf{MP\_GT} if $a$ is larger than $b$.

This function requires no additional memory and $O(N)$ time.

\subsubsection{mp\_cmp\_mag(mp\_int *a, mp\_int *b)}
Performs a \textbf{unsigned} comparison between $a$ and $b$ returning \textbf{MP\_GT} is $a$ is larger than $b$.  Note 
that this comparison is unsigned which means it will report, for example, $-5 > 3$.  By comparison mp\_cmp will 
report $-5 < 3$.

This function requires no additional memory and $O(N)$ time.

\subsubsection{mp\_add(mp\_int *a, mp\_int *b, mp\_int *c)}
Computes $c = a + b$ using signed arithmetic.  Handles the sign of the numbers which means it will subtract as 
required, e.g. $a + -b$ turns into $a - b$.

This function requires no additional memory and $O(N)$ time.

\subsubsection{mp\_sub(mp\_int *a, mp\_int *b, mp\_int *c)}
Computes $c = a - b$ using signed arithmetic.   Handles the sign of the numbers which means it will add as 
required, e.g. $a - -b$ turns into $a + b$.

This function requires no additional memory and $O(N)$ time.

\subsubsection{mp\_mul(mp\_int *a, mp\_int *b, mp\_int *c)}
Computes $c = a \cdot b$ using signed arithmetic.  Handles the sign of the numbers correctly which means it will 
correct the sign of the product as required, e.g. $a \cdot -b$ turns into $-ab$.

This function requires $O(N^2)$ time for small inputs and $O(N^{1.584})$ time for relatively large 
inputs (\textit{above the }KARATSUBA\_MUL\_CUTOFF \textit{value defined in bncore.c.}).  There is 
considerable overhead in the Karatsuba method which only pays off when the digit count is fairly high
(\textit{typically around 80}).  For small inputs the function requires $O(2N)$ memory, otherwise it
requires $O(6 \cdot \mbox{lg}(N) \cdot N)$ memory.


\subsubsection{mp\_sqr(mp\_int *a, mp\_int *b)}
Computes $b = a^2$ and fixes the sign of $b$ to be positive.

This function has a running time and memory requirement profile very similar to that of the 
mp\_mul function.  It is always faster and uses less memory for the larger inputs.

\subsubsection{mp\_div(mp\_int *a, mp\_int *b, mp\_int *c, mp\_int *d)}
Computes $c = \lfloor a/b \rfloor$ and $d \equiv a \mbox{ (mod }b\mbox{)}$.  The division is signed which means the sign
of the output is not always positive.  The sign of the remainder equals the sign of $a$ while the sign of the 
quotient equals the product of the ratios $(a/\vert a \vert) \cdot (b/\vert b \vert)$.  Both $c$ and $d$ can be 
optionally passed as NULL if the value is not desired.  For example, if you want only the quotient of $x/y$ then 
mp\_div(\&x, \&y, \&z, NULL) is acceptable.

This function requires $O(4 \cdot N)$ memory and $O(3 \cdot N^2)$ time.

\subsubsection{mp\_mod(mp\_int *a, mp\_int *b, mp\_int *c)}
Computes $c \equiv a \mbox{ (mod }b\mbox{)}$ but with the added condition that $0 \le c < b$.  That is a normal 
division is performed and if the remainder is negative $b$ is added to it.  Since adding $b$ modulo $b$ is equivalent
to adding zero ($0 \equiv b \mbox{ (mod }b\mbox{)}$) the result is accurate.  The results are undefined 
when $b \le 0$, in theory the routine will still give a properly congruent answer but it will not always be positive. 

This function requires $O(4 \cdot N)$ memory and $O(3 \cdot N^2)$ time.

\subsection{Number Theoretic Functions}

\subsubsection{mp\_addmod, mp\_submod, mp\_mulmod, mp\_sqrmod}
These functions take the time of their host function plus the time it takes to perform a division.  For example, 
mp\_addmod takes $O(N + 3 \cdot N^2)$ time.  Note that if you are performing many modular operations in a row with
the same modulus you should consider Barrett reductions.  

Also note that these functions use mp\_mod which means the result are guaranteed to be positive.

\subsubsection{mp\_invmod(mp\_int *a, mp\_int *b, mp\_int *c)}
This function will find $c = 1/a \mbox{ (mod }b\mbox{)}$ for any value of $a$ such that $(a, b) = 1$ and $b > 0$.  When
$b$ is odd a ``fast'' variant is used which finds the inverse twice as fast.  If no inverse is found (e.g. $(a, b) \ne 1$) then
the function returns \textbf{MP\_VAL} and the result in $c$ is undefined.

\subsubsection{mp\_gcd(mp\_int *a, mp\_int *b, mp\_int *c)}
Finds the greatest common divisor of both $a$ and $b$ and places the result in $c$.  Will work with either positive
or negative inputs.  

Functions requires no additional memory and approximately $O(N \cdot log(N))$ time.

\subsubsection{mp\_lcm(mp\_int *a, mp\_int *b, mp\_int *c)}
Finds the least common multiple of both $a$ and $b$ and places the result in $c$.  Will work with either positive
or negative inputs.  This is calculated by dividing the product of $a$ and $b$ by the greatest common divisor of 
both.  

Functions requires no additional memory and approximately $O(4 \cdot N^2)$ time.

\subsubsection{mp\_n\_root(mp\_int *a, mp\_digit b, mp\_int *c)}
Finds the $b$'th root of $a$ and stores it in $b$.  The roots are found such that $\vert c \vert^b \le \vert a \vert$.  
Uses the Newton approximation approach which means it converges in $O(log \beta^N)$ time to a final result.  Each iteration
requires $b$ multiplications and one division for a total work of $O(6N^2 \cdot log \beta^N) = O(6N^3 \cdot log \beta)$.

If the input $a$ is negative and $b$ is even the function returns \textbf{MP\_VAL}.  Otherwise the function will 
return a root that has a sign that agrees with the sign of $a$.

\subsubsection{mp\_jacobi(mp\_int *a, mp\_int *n, int *c)}
Computes $c = \left ( {a \over n} \right )$ or the Jacobi function of $(a, n)$ and stores the result in an integer addressed
by $c$.  Since the result of the Jacobi function $\left ( {a \over n} \right ) \in \lbrace -1, 0, 1 \rbrace$ it seemed
natural to store the result in a simple C style \textbf{int}.  If $n$ is prime then the Jacobi function produces
the same results as the Legendre function\footnote{Source: Handbook of Applied Cryptography, pp. 73}.  This means if
$n$ is prime then $\left ( {a \over n} \right )$ is equal to $1$ if $a$ is a quadratic residue modulo $n$ or $-1$ if 
it is not.

\subsubsection{mp\_exptmod(mp\_int *a, mp\_int *b, mp\_int *c, mp\_int *d)}
Computes $d \equiv a^b \mbox{ (mod }c\mbox{)}$ using a sliding window $k$-ary exponentiation algorithm.  For an $\alpha$-bit
exponent it performs $\alpha$ squarings and at most $\lfloor \alpha/k \rfloor + 2^{k-1}$ multiplications.  The value of $k$ is
chosen to minimize the number of multiplications required for a given value of $\alpha$.  Barrett, Montgomery or
Dimminished-Radix reductions are used to reduce the squared or multiplied temporary results modulo $c$.

\subsection{Fast Modular Reductions}

A modular reduction of $a \mbox{ (mod }b\mbox{)}$ means to divide $a$ by $b$ and obtain the remainder.  
Typically modular reductions are popular in public key cryptography algorithms such as RSA, 
Diffie-Hellman and Elliptic Curve.  Modular reductions are also a large portion of modular exponentiation 
(e.g. $a^b \mbox{ (mod }c\mbox{)}$).  

In a simplistic sense a normal integer division could be used to compute reduction.  Division is by far
the most complicated of routines in terms of the work required.  As a result it is desirable to avoid
division as much as possible.  This is evident in quite a few fields in computing.  For example, often in
signal analysis uses multiplication by the reciprocal to approximate divisions.  Number theory is no
different.

In most cases for the reduction of $a$ modulo $b$ the integer $a$ will be limited to the range 
$0 \le a \le b^2$ which led to the invention of specialized algorithms to do the work.

The first algorithm is the most generic and is called the Barrett reduction.  When the input is of the 
limited form (e.g. $0 \le a \le b^2$) Barrett reduction is numerically equivalent to a full integer
division with remainder.  For a $n$-digit value $b$ the Barrett reduction requires approximately $2n^2$
multiplications.

The second algorithm is the Montgomery reduction.  It is slightly different since the result is not
numerically equivalent to a standard integer division with remainder.  Also this algorithm only works for
odd moduli.  The final result can be converted easily back to the desired for which makes the reduction 
technique useful for algorithms where only the final output is desired.  For a $n$-digit value $b$ the 
Montgomery reduction requires approximately $n^2 + n$ multiplications, about half as many as the 
Barrett algorithm.  

The third algorithm is the Diminished Radix ``DR'' reduction.  It is a highly optimized reduction algorithm
suitable only for a limited set of moduli.  For the specific moduli it is numerically equivalent to
integer division with remainder.  For a $n$-digit value $b$ the DR reduction rquires exactly $n$
multiplications which is considerably faster than either of the two previous algorithms.

All three algorithms are automatically used in the modular exponentiation function mp\_exptmod() when 
appropriate moduli are detected.

\begin{figure}[here]
\begin{small}
\begin{center}
\begin{tabular}{|c|c|l|}
\hline \textbf{Algorithm} & \textbf{Multiplications} & \textbf{Limitations} \\
 Barrett Reduction  & $2n^2$ & Any modulus. \\
 Montgomery Reduction & $n^2 + n$ & Any odd modulus. \\
 DR Reduction & $n$ & Moduli of the form  $p = \beta^k - p'$.\\
\hline
\end{tabular}
\caption{Summary of reduction techniques.}
\end{center}
\end{small}
\end{figure}

\subsubsection{mp\_reduce(mp\_int *a, mp\_int *b, mp\_int *c)}
Computes a Barrett reduction in-place of $a$ modulo $b$ with respect to $c$.  In essence it computes 
$a \mbox{ (mod }b\mbox{)}$ provided $0 \le a \le b^2$.  The value of $c$ is precomputed with the 
function mp\_reduce\_setup().  The modulus $b$ must be larger than zero.

This reduction function is much faster than simply calling mp\_mod() (\textit{Which simply uses mp\_div() anyways}) and is
desirable where ever an appropriate reduction is desired.  

The Barrett reduction function has been optimized to use partial multipliers which means compared to MPI it performs
have the number of single precision multipliers (\textit{provided they have the same size digits}).  The partial
multipliers (\textit{one of which is shared with mp\_mul}) both have baseline and comba variants.  Barrett reduction 
can reduce a number modulo a $n-$digit modulus with approximately $2n^2$ single precision multiplications.  

Consider the following snippet (from a BBS generator) using the more traditional approach:

\begin{small}
\begin{verbatim}
   mp_int modulus, n;
   unsigned char buf[128];
   int ix, err;
   
   /* ... init code ..., e.g. init modulus and n */
   /* now output 128 bytes */
   for (ix = 0; ix < 128; ix++) { 
       if ((err = mp_sqrmod(&n, &modulus, &n)) != MP_OKAY) {
          printf("Err: %d\n", err);
          exit(EXIT_FAILURE);
       }
       buf[ix] = n->dp[0] & 255;
   }
\end{verbatim}
\end{small}

And now consider the same function using Barrett reductions:

\begin{small}
\begin{verbatim}
   mp_int modulus, n, mp;
   unsigned char buf[128];
   int ix, err;
   
   /* ... init code ... e.g. modulus and n */
   
   /* now setup mp which is the Barrett param */
   if ((err = mp_reduce_setup(&mp, &modulus)) != MP_OKAY) {
      printf("Err: %d\n", err);
      exit(EXIT_FAILURE);
   }
   /* now output 128 bytes */
   for (ix = 0; ix < 128; ix++) {
      /* square n */
      if ((err = mp_sqr(&n, &n)) != MP_OKAY) {
         printf("Err: %d\n", err);
         exit(EXIT_FAILURE);
      }
      /* now reduce the square modulo modulus */
      if ((err = mp_reduce(&n, &modulus, &mp)) != MP_OKAY) {
         printf("Err: %d\n", err);
         exit(EXIT_FAILURE);
      }
      buf[ix] = n->dp[0] & 255;
   }
\end{verbatim}	
\end{small}

Both routines will produce the same output provided the same initial values of $modulus$ and $n$.  The Barrett
method seems like more work but the optimization stems from the use of the Barrett reduction instead of the normal
integer division.

\subsubsection{mp\_montgomery\_reduce(mp\_int *a, mp\_int *m, mp\_digit mp)}
Computes a Montgomery reduction in-place of $a$ modulo $b$ with respect to $mp$.  If $b$ is some $n-$digit modulus then
$R = \beta^{n+1}$.  The result of this function is $aR^{-1} \mbox{ (mod }b\mbox{)}$ provided that $0 \le a \le b^2$.
The value of $mp$ is precomputed with the function mp\_montgomery\_setup().  The modulus $b$ must be odd and larger
than zero.  

The Montgomery reduction comes in two variants.  A standard baseline and a fast comba method.  The baseline routine
is in fact slower than the Barrett reductions, however, the comba routine is much faster.  Montomgery reduction can 
reduce a number modulo a $n-$digit modulus with approximately $n^2 + n$ single precision multiplications.  Compared
to Barrett reductions the montgomery reduction requires half as many multiplications as $n \rightarrow \infty$.  

Note that the final result of a Montgomery reduction is not just the value reduced modulo $b$.  You have to multiply
by $R$ modulo $b$ to get the real result.  At first that may not seem like such a worthwhile routine, however, the
exptmod function can be made to take advantage of this such that only one normalization at the end is required.

This stems from the fact that if $a \rightarrow aR^{-1}$ through Montgomery reduction and if $a = vR$ and $b = uR$ then
$a^2 \rightarrow v^2R^2R^{-1} \equiv v^2R$ and $ab \rightarrow uvRRR^{-1} \equiv uvR$.  The next useful observation is 
that through the reduction $a \rightarrow vRR^{-1} \equiv v$ which means given a final result it can be normalized with
a single reduction.  Now a series of complicated modular operations can be optimized if all the variables are initially
multiplied by $R$ then the final result normalized by performing an extra reduction.

If many variables are to be normalized the simplest method to setup the variables is to first compute $\hat x \equiv R^2 \mbox{ mod }m$.
Now all the variables in the system can be multiplied by $\hat x$ and reduced with Montgomery reduction.  This means that
two long divisions would be required to setup $\hat x$ and a multiplication followed by reduction for each variable.  

A very useful observation is that multiplying by $R = \beta^n$ amounts to performing a left shift by $n$ positions which
requires no single precision multiplications.

\subsubsection{mp\_dr\_reduce(mp\_int *a, mp\_int *b, mp\_digit mp)}
Computes the Diminished-Radix reduction of $a$ in place modulo $b$ with respect to $mp$.  $a$ must be in the range 
$0 \le a \le b^2$ and $mp$ must be precomputed with the function mp\_dr\_setup().

This reduction technique performs the reduction with $n$ multiplications and is much faster than either of the previous
reduction methods.  Essentially it is very much like the Montgomery reduction except it is particularly optimized for
specific types of moduli.  The moduli must be of the form $p = \beta^k - p'$ where $0 \le p' < \beta$ for $k \ge 2$.  
This algorithm is suitable for several applications such as Diffie-Hellman public key cryptsystems where the prime $p$ is 
of this form.

In appendix A several ``safe'' primes of various sizes are provided.  These primes are DR moduli and of the form 
$p = 2q + 1$ where both $p$ and $q$ are prime.  A trivial observation is that $g = 4$ will be a generator for all of them
since the order of the multiplicative sub-group is at most $2q$.  Since $2^2 \ne 1$ that means $4^q \equiv 2^{2q} \equiv 1$ 
and that $g = 4$ is a generator of order $q$.

These moduli can be used to construct a Diffie-Hellman public key cryptosystem.  Since the moduli are of the
DR form the modular exponentiation steps will be efficient.

\subsection{Primality Testing and Generation}

\subsubsection{mp\_prime\_is\_divisible(mp\_int *a, int *result)}
Determines if $a$ is divisible by any of the first 256 primes.  Sets $result$ to $1$ if true or $0$ 
otherwise.  Also will set $result$ to $1$ if $a$ is equal to one of the first 256 primes.  

\subsubsection{mp\_prime\_fermat(mp\_int *a, mp\_int *b, int *result)}
Determines if $b$ is a witness to the compositeness of $a$ using the Fermat test.  Essentially this
computes $b^a \mbox{ (mod }a\mbox{)}$ and compares it to $b$.  If they match $result$ is set
to $1$ otherwise it is set to $0$.  If $a$ is prime and $1 < b < a$ then this function will set 
$result$ to $1$ with a probability of one.  If $a$ is composite then this function will set 
$result$ to $1$ with a probability of no more than $1 \over 2$.  

If this function is repeated $t$ times with different bases $b$ then the probability of a false positive
is no more than $2^{-t}$.

\subsubsection{mp\_prime\_miller\_rabin(mp\_int *a, mp\_int *b, int *result)}
Determines if $b$ is a witness to the compositeness of $a$ using the Miller-Rabin test.  This test
works much (\textit{on an abstract level}) the same as the Fermat test except is more robust.  The
set of pseudo-primes to any given base for the Miller-Rabin test is a proper subset of the pseudo-primes
for the Fermat test.  

If $a$ is prime and $1 < b < a$ then this function will always set $result$ to $1$.  If $a$ is composite
the trivial bound of error is $1 \over 4$.  However, according to HAC\footnote{Handbook of Applied
Cryptography, Chapter 4, Section 4, pp. 147, Fact 4.48.} the following bounds are 
known.  For a test of $t$ trials on a $k$-bit number the probability $P_{k,t}$ of error is given as
follows.

\begin{enumerate}
\item $P_{k,1} < k^24^{2 - \sqrt{k}}$ for $k \ge 2$
\item $P_{k,t} < k^{3/2}2^tt^{-1/2}4^{2-\sqrt{tk}}$ for $(t = 2, k \ge 88)$ or $(3 \le t \le k/9, k \ge 21)$.
\item $P_{k,t} < {7 \over 20}k2^{-5t} + {1 \over 7}k^{15/4}2^{-k/2-2t} + 12k2^{-k/4-3t}$ for $k/9 \le t \le k/4, k \ge 21$.
\item $P_{k,t} < {1 \over 7}k^{15/4}2^{-k/2 - 2t}$  for $t \ge k/4, k \ge 21$.
\end{enumerate}

For instance, $P_{1024,1}$ which indicates the probability of failure of one test with a 1024-bit candidate 
is no more than $2^{-40}$.  However, ideally at least a couple of trials should be used.  In LibTomCrypt
for instance eight tests are used.  In this case $P_{1024,8}$ falls under the second rule which leads
to a probability of failure of no more than $2^{-155.52}$.

\begin{figure}[here]
\begin{small}
\begin{center}
\begin{tabular}{|c|c|c|c|c|c|c|}
\hline \textbf{Size (k)} & \textbf{$t = 3$} & \textbf{$t = 4$} & \textbf{$t = 5$} & \textbf{$t = 6$} & \textbf{$t = 7$} & \textbf{$t = 8$}\\
\hline 512  & -58 & -70 & -79 & -88 & -96 & -104 \\
\hline 768  & -75 & -89 & -101 & -112 & -122 & -131\\
\hline 1024 & -89 & -106 & -120 & -133 & -144 & -155 \\
\hline 1280 & -102 & -120 & -136 & -151 & -164 & -176 \\
\hline 1536 & -113 & -133 & -151 & -167 & -181 & -195 \\
\hline 1792 & -124 & -146 & -165 & -182 & -198 & -212 \\
\hline 2048 & -134 & -157 & -178 & -196 & -213 & -228\\
\hline
\end{tabular}
\end{center}
\end{small}
\caption{Probability of error for a given random candidate of $k$ bits with $t$ trials.  Denoted as 
log$_2(P_{k,t})$. }
\end{figure}

\subsubsection{mp\_prime\_is\_prime(mp\_int *a, int t, int *result)}
This function determines if $a$ is probably prime by first performing trial division by the first 256 
primes and then $t$ rounds of Miller-Rabin using the first $t$ primes as bases.  If $a$ is prime this
function will always set $result$ to $1$.  If $a$ is composite then it will almost always set $result$
to $0$.  The probability of error is given in figure two.

\subsubsection{mp\_prime\_next\_prime(mp\_int *a, int t, int bbs\_style)}
This function will find the next prime \textbf{after} $a$ by using trial division and $t$ trials of 
Miller-Rabin.  If $bbs\_style$ is set to one than $a$ will be the next prime such that $a \equiv 3 \mbox{ (mod }4\mbox{)}$ 
which is useful for certain algorithms.  Otherwise, $a$ will be the next prime greater than the initial input
value and may be $\lbrace 1, 3 \rbrace \equiv a \mbox{ (mod }4\mbox{)}$.  

\section{Timing Analysis}

\subsection{Digit Size}
The first major constribution to the time savings is the fact that 28 bits are stored per digit instead of the MPI 
defualt of 16.  This means in many of the algorithms the savings can be considerable.  Consider a baseline multiplier 
with a 1024-bit input.  With MPI the input would be 64 16-bit digits whereas in LibTomMath it would be 37 28-bit digits.
A savings of $64^2 - 37^2 = 2727$ single precision multiplications.  

\subsection{Multiplication Algorithms}
For most inputs a typical baseline $O(n^2)$ multiplier is used which is similar to that of MPI.  There are two variants 
of the baseline multiplier.  The normal and the fast comba variant.  The normal baseline multiplier is the exact same as 
the algorithm from MPI.  The fast comba baseline multiplier is optimized for cases where the number of input digits $N$ 
is less than or equal to $2^{w}/\beta^2$.  Where $w$ is the number of bits in a \textbf{mp\_word} or simply $lg(\beta)$.
By default a mp\_word is 64-bits which means $N \le 256$ is allowed which represents numbers upto $7,168$ bits.  However,
since the Karatsuba multiplier (discussed below) will kick in before that size the slower baseline algorithm (that MPI
uses) should never really be used in a default configuration.  

The fast comba baseline multiplier is optimized by removing the carry operations from the inner loop.  This is often 
referred to as the ``comba'' method since it computes the products a columns first then figures out the carries.  To
accomodate this the result of the inner multiplications must be stored in words large enough not to lose the carry bits.  
This is why there is a limit of $2^{w}/\beta^2$ digits in the input.  This optimization has the effect of making a 
very simple and efficient inner loop.

\subsubsection{Karatsuba Multiplier}
For large inputs, typically 80 digits\footnote{By default that is 2240-bits or more.} or more the Karatsuba multiplication
method is used.  This method has significant overhead but an asymptotic running time of $O(n^{1.584})$ which means for 
fairly large inputs this method is faster than the baseline (or comba) algorithm.  The Karatsuba implementation is 
recursive which means for extremely large inputs they will benefit from the algorithm.

The algorithm is based on the observation that if 

\begin{eqnarray}
x = x_0 + x_1\beta \nonumber \\
y = y_0 + y_1\beta
\end{eqnarray}

Where $x_0, x_1, y_0, y_1$ are half the size of their respective summand than 

\begin{equation}
x \cdot y = x_1y_1\beta^2 + ((x_1 - y_1)(x_0 - y_0) + x_0y_0 + x_1y_1)\beta + x_0y_0
\end{equation}

It is trivial that from this only three products have to be produced: $x_0y_0, x_1y_1, (x_1-y_1)(x_0-y_0)$ which
are all of half size numbers.  A multiplication of two half size numbers requires only $1 \over 4$ of the
original work which means with no recursion the Karatsuba algorithm achieves a running time of ${3n^2}\over 4$.  
The routine provided does recursion which is where the $O(n^{1.584})$ work factor comes from.

The multiplication by $\beta$ and $\beta^2$ amount to digit shift operations.  
The extra overhead in the Karatsuba method comes from extracting the half size numbers $x_0, x_1, y_0, y_1$ and
performing the various smaller calculations.  

The library has been fairly optimized to extract the digits using hard-coded routines instead of the hire
level functions however there is still significant overhead to optimize away.

MPI only implements the slower baseline multiplier where carries are dealt with in the inner loop.  As a result even at
smaller numbers (below the Karatsuba cutoff) the LibTomMath multipliers are faster.

\subsection{Squaring Algorithms}

Similar to the multiplication algorithms there are two baseline squaring algorithms.  Both have an asymptotic 
running time of $O((t^2 + t)/2)$.  The normal baseline squaring is the same from MPI and the fast method is 
a ``comba'' squaring algorithm.  The comba method is used if the number of digits $N$ is less than 
$2^{w-1}/\beta^2$ which by default covers numbers upto $3,584$ bits.  

There is also a Karatsuba squaring method which achieves a running time of $O(n^{1.584})$ after considerably large
inputs.

MPI only implements the slower baseline squaring algorithm.  As a result LibTomMath is considerably faster at squaring
than MPI is.

\subsection{Exponentiation Algorithms}

LibTomMath implements a sliding window $k$-ary left to right exponentiation algorithm.  For a given exponent size $L$ an
appropriate window size $k$ is chosen.  There are always at most $L$ modular squarings and $\lfloor L/k \rfloor$ modular
multiplications.   The $k$-ary method works by precomputing values $g(x) = b^x$ for $2^{k-1} \le x < 2^k$ and a given base 
$b$.  Then the multiplications are grouped in windows of $k$ bits.  The sliding window technique has the benefit 
that it can skip multiplications if there are zero bits following or preceding a window.  Consider the exponent 
$e = 11110001_2$ if $k = 2$ then there will be a two squarings, a multiplication of $g(3)$, two squarings, a multiplication
of $g(3)$, four squarings and and a multiplication by $g(1)$.  In total there are 8 squarings and 3 multiplications.

MPI uses a binary square-multiply method for exponentiation.  For the same exponent $e = 11110001_2$ it would have had to
perform 8 squarings and 5 multiplications.  There is a precomputation phase for the method LibTomMath uses but it 
generally cuts down considerably on the number of multiplications.  Consider a 512-bit exponent.  The worst case for the 
LibTomMath method results in 512 squarings and 124 multiplications.  The MPI method would have 512 squarings 
and 512 multiplications.  

Randomly the most probable event is that every $2k^2$ bits another multiplication is saved via the 
sliding-window technique on top of the savings the $k$-ary method provides.  This stems from the fact that each window
has a probability of $2^{-1}$ of being delayed by one bit.  In reality the savings can be much more when the exponent
has an abundance of zero bits.  

Both LibTomMath and MPI use Barrett reduction instead of division to reduce the numbers modulo the modulus given.
However, LibTomMath can take advantage of the fact that the multiplications required within the Barrett reduction
do not have to give full precision.  As a result the reduction step is much faster and just as accurate.  The LibTomMath 
code will automatically determine at run-time (e.g. when its called) whether the faster multiplier can be used.  The
faster multipliers have also been optimized into the two variants (baseline and comba baseline).

LibTomMath also has a variant of the exptmod function that uses Montgomery or Diminished-Radix reductions instead of 
Barrett reductions which are faster.  The code will automatically detect when the Montgomery version can be used 
(\textit{Requires the modulus to be odd and below the MONTGOMERY\_EXPT\_CUTOFF size}).  The Montgomery routine is 
essentially a copy of the Barrett exponentiation routine except it uses Montgomery reduction.

As a result of all these changes exponentiation in LibTomMath is much faster than compared to MPI.  On most ALU-strong
processors (AMD Athlon for instance) exponentiation in LibTomMath is often more then ten times faster than MPI.

\newpage
\section*{Appendix A -- DR Safe Prime Moduli}
These are safe primes suitable for the DR reduction techniques.

\begin{small}
\begin{verbatim}
224-bit prime:
p == 26959946667150639794667015087019630673637144422540572481103341844143

532-bit prime:
p == 14059105607947488696282932836518693308967803494693489478439861164411
     99243959839959474700214407465892859350284572975279726002583142341968
     6528151609940203368691747

784-bit prime:
p == 10174582569701926077392351975587856746131528201775982910760891436407
     52752352543956225804474009941755789631639189671820136396606697711084
     75957692810857098847138903161308502419410142185759152435680068435915
     159402496058513611411688900243039
     
1036-bit prime:
p == 73633510803960459580592340614718453088992337057476877219196961242207
     30400993319449915739231125812675425079864519532271929704028930638504
     85730703075899286013451337291468249027691733891486704001513279827771
     74018362916106519487472796251714810077522836342108369176406547759082
     3919364012917984605619526140821798437127

1540-bit prime:
p == 38564998830736521417281865696453025806593491967131023221754800625044
     11826546885121070536038571753679461518026049420807660579867166071933
     31995138078062523944232834134301060035963325132466829039948295286901
     98205120921557533726473585751382193953592127439965050261476810842071
     57368450587885458870662348457392592590350574754547108886771218500413
     52012892734056144158994382765356263460989042410208779740029161680999
     51885406379295536200413493190419727789712076165162175783
     
2072-bit prime:
p == 54218939133169617266167044061918053674999416641599333415160174539219
     34845902966009796023786766248081296137779934662422030250545736925626
     89251250471628358318743978285860720148446448885701001277560572526947
     61939255157449083928645845499448866574499182283776991809511712954641
     41244487770339412235658314203908468644295047744779491537946899487476
     80362212954278693335653935890352619041936727463717926744868338358149
     56836864340303776864961677852601361049369618605589931826833943267154
     13281957242613296066998310166663594408748431030206661065682224010477
     20269951530296879490444224546654729111504346660859907296364097126834
     834235287147
\end{verbatim}
\newpage
\begin{verbatim}
3080-bit prime:
p == 14872591348147092640920326485259710388958656451489011805853404549855
     24155135260217788758027400478312256339496385275012465661575576202252
     06314569873207988029466422057976484876770407676185319721656326266004
     66027039730507982182461708359620055985616697068444694474354610925422
     65792444947706769615695252256130901271870341005768912974433684521436
     21126335809752272646208391793909176002665892575707673348417320292714
     14414925737999142402226287954056239531091315945236233530448983394814
     94120112723445689647986475279242446083151413667587008191682564376412
     34796414611389856588668313940700594138366932599747507691048808666325
     63356891811579575714450674901879395531659037735542902605310091218790
     44170766615232300936675369451260747671432073394867530820527479172464
     10644245072764022650374658634027981631882139521072626829153564850619
     07146160831634031899433344310568760382865303657571873671474460048559
     12033137386225053275419626102417236133948503

4116-bit prime:
p == 10951211157166778028568112903923951285881685924091094949001780089679
     55253005183831872715423151551999734857184538199864469605657805519106
     71752965504405483319768745978263629725521974299473675154181526972794
     07518606702687749033402960400061140139713092570283328496790968248002
     50742691718610670812374272414086863715763724622797509437062518082383
     05605014462496277630214789052124947706021514827516368830127584715531
     60422794055576326393660668474428614221648326558746558242215778499288
     63023018366835675399949740429332468186340518172487073360822220449055
     34058256846156864525995487330361695377639385317484513208112197632746
     27403549307444874296172025850155107442985301015477068215901887335158
     80733527449780963163909830077616357506845523215289297624086914545378
     51108253422962011656326016849452390656670941816601111275452976618355
     45793212249409511773940884655967126200762400673705890369240247283750
     76210477267488679008016579588696191194060127319035195370137160936882
     40224439969917201783514453748848639690614421772002899286394128821718
     53539149915834004216827510006035966557909908155251261543943446413363
     97793791497068253936771017031980867706707490224041075826337383538651
     82549367950377193483609465580277633166426163174014828176348776585274
     6577808019633679
\end{verbatim}
\end{small}

%\newpage
%\section*{Appendix B - Function Quick Sheet}

%The following table gives a quick summary of the functions provided within LibTomMath.

%\begin{flushleft}
%\begin{tiny}
%\begin{tabular}{|l|l|l|}
%\hline \textbf{Function Name} & \textbf{Purpose} & \textbf{Notes} \\
%\hline mp\_init(mp\_int *a) & Initializes a mp\_int & Allocates runtime memory required for an integer \\
%\hline mp\_clear(mp\_int *a) & Frees the ram used by an mp\_int & \\
%\hline mp\_exch(mp\_int *a, mp\_int *b) & Swaps two mp\_int structures contents & \\
%\hline mp\_shrink(mp\_int *a) & Frees unused memory & The mp\_int is still valid and not cleared. \\
%\hline mp\_grow(mp\_int *a, int size) & Ensures that a has at least $size$ digits allocated & \\
%\hline mp\_init\_size(mp\_int a, int size) & Inits and ensures it has at least $size$ digits & \\
%\hline &&\\
%\hline mp\_zero(mp\_int *a) & $a \leftarrow 0$ & \\
%\hline mp\_set(mp\_int *a, mp\_digit b) & $a \leftarrow b$ & \\
%\hline mp\_set\_int(mp\_int *a, unsigned long b) & $a \leftarrow b$ & Only reads upto 32 bits from $b$ \\
%\hline &&\\
%\hline mp\_rshd(mp\_int *a, int b) & $a \leftarrow a/\beta^b$ & \\
%\hline mp\_lshd(mp\_int *a, int b) & $a \leftarrow a \cdot \beta^b$ &\\
%\hline mp\_div\_2d(mp\_int *a, int b, mp\_int *c, mp\_int *d) & &\\
%\hline
%\end{tabular}
%\end{tiny}
%\end{flushleft}

\end{document}


\end{document}
